\documentclass[a4paper, DIV19,8pt]{scrbook}
\usepackage[ngerman]{babel}
\usepackage[OT2,T1]{fontenc}
\DeclareSymbolFont{cyrletters}{OT2}{wncyr}{m}{n}
\DeclareMathSymbol{\Sha}{\mathalpha}{cyrletters}{"58}
\DeclareMathSymbol{\sha}{\mathalpha}{cyrletters}{"57}
\usepackage[utf8]{inputenc}
\usepackage[table]{xcolor}
\usepackage[fleqn]{amsmath}
\usepackage{amsfonts}
\usepackage{amssymb}
\usepackage{amstext}
\usepackage{mathtools}
\usepackage[utopia]{mathdesign}
%\usepackage{listings}
\usepackage{tabularx}
\usepackage{mdwlist}
\usepackage{booktabs}
\usepackage{framed}
\usepackage{tikz}
\usepackage{pdfpages}
\usepackage{esvect}
\usepackage{trfsigns}
%\usepackage{pstricks}
\usepackage[locale=DE]{siunitx}
\usetikzlibrary{decorations.pathreplacing}

\usepackage[inner=2.5cm,outer=1.5cm, top=1cm,bottom=1cm,footskip=.5cm]{geometry}

%Si einheiten durch einen Bruch ersetzen
%\sisetup{per=frac, fraction=nice}
\allowdisplaybreaks
\setlength{\intextsep}{0mm}
\setlength{\textfloatsep}{0mm}
\setlength{\floatsep}{0mm}
\definecolor{lgray}{rgb}{0.7,0.7,0.7}
%Standart
\definecolor{shadecolor}{rgb}{0.8,0.8,0.8}
\definecolor{gray}{rgb}{0.5,0.5,0.5}


%Daniel der Betrüger
%\definecolor{shadecolor}{rgb}{0.5,0.82,0.945}
%\definecolor{gray}{rgb}{0.7,0.2,0.2}


%\lstset{basicstyle=\small\ttfamily,backgroundcolor=\color{lgray}}
%\newcommand{\bla}[1]{\noindent\quad\textbf{#1}}
\newcommand{\bla}[1]{\subsubsection{#1}}


\newcommand{\Mseite}[1]{\begin{empheq}[box=\fcolorbox{gray}{lgray}]{equation}#1\end{empheq}}

%Abstand zwischen Formeln
\setlength{\abovedisplayskip}{3mm}
\setlength{\belowdisplayskip}{3mm}
%Größe einer Shaded box
\setlength{\FrameSep}{1pt}
\setlength{\parindent}{0pt}

\newenvironment{merkbox}{\begin{minipage}{\linewidth}}{\end{minipage}}

%Box Centriert
\newenvironment{boxleftc}{\begin{minipage}{.5\linewidth}\noindent}{\end{minipage}}
\newenvironment{boxrightc}{\begin{minipage}{.5\linewidth}\noindent}{\end{minipage}}
\newenvironment{boxrightshadedc}{\begin{minipage}{.5\linewidth}\begin{shaded}\noindent}{\end{shaded}\end{minipage}}

\newenvironment{boxshaded}{\begin{minipage}{\linewidth}\begin{shaded}\noindent}{\end{shaded}\end{minipage}}


%Box viertel
\newenvironment{boxleftv}{\begin{minipage}{.25\linewidth}\noindent}{\end{minipage}}
\newenvironment{boxrightv}{\begin{minipage}{.25\linewidth}\noindent}{\end{minipage}}
\newenvironment{boxrightshadedv}{\begin{minipage}{.25\linewidth}\begin{shaded}\noindent}{\end{shaded}\end{minipage}}


%Box Rechts
\newenvironment{boxleft}{\begin{minipage}{.35\linewidth}\noindent}{\end{minipage}}
\newenvironment{boxright}{\begin{minipage}{.65\linewidth}\noindent}{\end{minipage}}
\newenvironment{boxrightshaded}{\begin{minipage}{.65\linewidth}\begin{shaded}\noindent}{\end{shaded}\end{minipage}}

%Einheitenbeschreibung
\newcommand{\des}[3][1]{$\left[#2\right]=\si{#1}$: {\small\textcolor{gray}{ #3}}}
\newcommand{\destext}[1]{{\small\textcolor{gray}{ #1}}}

\newcommand{\hebox}[1]{#1}
\newcommand{\heboxc}[1]{#1}

%Differential
\newcommand*\diff{\mathop{}\!\mathrm{d}}
\newcommand*\grad{\mathop{}\!\mathrm{grad}}


\DeclareSIUnit{\year}{a}

\begin{document}
\title{Formelsammlung für Alles}
\author{Matthias Springstein}
\maketitle
\tableofcontents

%\chapter{Mathe}
\section{Grundlagen}
\subsection{Mengen}
\bla{Mengen Darstellung}

\noindent\begin{tabularx}{\textwidth}{lX}

\toprule
 Schreibweise & Bedeutung \\
\midrule
$a \in M:$ & a ist ein Element von M \\
$a \not \in M:$ & a ist kein Element von M \\
$M=\{x | x \text{ Eigenschaften},\ldots\}$ & Beschreibende Darstellung \\
$M=\{a_1,a_2\ldots,a_n\}$ &Aufzählende Darstellung(endlich) \\
$M=\{a_1,a_2\ldots\}$ & Aufzählende Darstellung(unendlich) \\
$M=\{\}$ &Leere Menge \\
$A\subset B$ & A ist eine \emph{Teilmenge} von B. A heißt \emph{Untermenge} und B \emph{Obermenge}\\
$A=B$ & A und B sind gleich, d.h. jedes Element von A ist auch in B vorhanden und umgekehrt\\
\bottomrule
\end{tabularx}

\bla{Mengen Operationen}

\noindent\begin{tabularx}{\textwidth}{lX}
\toprule
 Schreibweise & Bedeutung \\
\midrule
$A\cap B=\{x|x\in A \text{ und } x\in B$ & \emph{Schnittmenge} zweier Mengen \\
$A\cup B=\{x|x\in A \text{ oder } x\in B\}$ & \emph{Vereinigungsmenge} zweier Mengen \\
$A\setminus B=\{x|x\in A \text{ und } x\not \in B\}$ & \emph{Differenz- oder Restmenge} zweier Mengen \\
\bottomrule
\end{tabularx}


\subsection{Intervalle}

\begin{tabularx}{\textwidth}{p{4cm}X}
\toprule
 Beispiel & Beschreibung \\
\midrule
\noindent
$[a,b]={x|a\leq x\leq b}$ & abgeschlossene Intervalle \\
$[a,b)={x|a\leq x < b}$ & halboffene Intervall\\
$(a,b]={x|a< x \leq b}$ & halboffene Intervall\\
$(a,b)={x|a<x<b}$ & offenes Intervall\\
\bottomrule
\end{tabularx}

\subsection{Rechnengesetze}

 \bla{Operationen mit Natürlichen Zahlen}
\begin{tabularx}{\textwidth}{p{3cm}X}
\toprule
 Beispiel & Beschreibung \\
\midrule
{\begin{align*}60 & =2^2\cdot3^1\cdot5^1 \\ 70 & =2^3\cdot3^2\\\text{ggt} & = 2^2\cdot3^1\end{align*}} & Zerlegung der Faktoren in ihre Primfaktoren und dann bildet man das Produkt aus denn höchsten Potenzen die alle Faktoren gemeinsam haben. \\
{\begin{align*}60 & =2^2\cdot3^1\cdot5^1 \\ 70 & =2^3\cdot3^2\\\text{kgV} & = 2^3\cdot3^2\cdot5^1\end{align*}}  &
Zerlegung der Faktoren in ihre Primfaktoren und dann bildet man das Produkt aus denn höchsten Potenzen die in mindestens einen Faktoren auftreten. \\
\bottomrule
\end{tabularx}

\bla{Kommutativgesetz}
\begin{shaded}
\begin{equation} \begin{split} a+b&=b+a\\a\cdot b&=b\cdot a \end{split} \end{equation} 
\end{shaded}
\bla{Assoziativgesetz}
\begin{shaded}
 

\begin{equation} \begin{split}a+(b+c) &=(a+b)+c\\a\cdot(b\cdot c)&=(a\cdot b)\cdot c\end{split} \end{equation}
\end{shaded}
\bla{Distributivgesetz}
 \begin{shaded}
\begin{equation} \begin{split}a\cdot(b+c)&=a\cdot b+a\cdot c\end{split}  \end{equation}
\end{shaded}

\subsection{Bruchrechnung}
Ein Bruch $a/b$ heißt \emph{echte}, wenn $|a|<|b|$ ist, sonst \emph{unecht}.

\bla{Addition und Subtraktion zweier Brüche}
\begin{shaded}
\begin{equation}
 \frac{a}{b}\pm\frac{c}{d}=\frac{a\cdot d \pm b\cdot c}{b\cdot d}
\end{equation}
\end{shaded}

\bla{Multiplikation zweier Brüche}
\begin{shaded}
\begin{equation}
 \frac{a}{b}\cdot\frac{c}{d}=\frac{a\cdot c}{b\cdot d}
\end{equation}
\end{shaded}
\bla{Division zweier Brüche}
\begin{shaded}
\begin{equation}
 \frac{a}{b}\div\frac{c}{d}=\frac{a\cdot d}{b\cdot c}
\end{equation}
\end{shaded}

\subsection{Potenzen}
Eine Potenz $a^n$ ist ein Produkt aus n gleichen Faktoren a:
\begin{shaded}
\begin{equation}
 a^n=a\cdot a \cdot a \ldots a
\end{equation}
$a:\text{ Basis}$ $n:\text{ Exponent}$
\end{shaded}
\bla{Rechenregeln}
\begin{shaded}
\begin{subequations}
\begin{equation}
 a^m*a^n=a^{m+n}
\end{equation}
\begin{equation}
\frac{a^m}{a^n}=a^{m-n}
\end{equation} 
\begin{equation}
 \left(a^m\right)^n=a^{m\cdot n}
\end{equation} 
\begin{equation}
 a^n\cdot b^n =\left(a\cdot b\right)^n
\end{equation} 
\begin{equation}
 \frac{a^n}{b^n}=\left(\frac{a}{b}\right)^n
\end{equation}
\end{subequations}
\end{shaded}
\subsection{Wurzeln}
Wurzelziehen ist die Umkehrfunktion des Potenzieren
\begin{shaded}
 \begin{equation}
\sqrt[n]{a}=a^{\left(\frac{1}{n}\right)}
\end{equation}
$a:\text{ Radikand}$ $n:\text{ Wurzelexponent}$  
\end{shaded}
\bla{Rechenregeln}
\begin{shaded}
\begin{subequations}
\begin{equation}
 \sqrt[n]{a^m}=a^{\left(\frac{m}{n}\right)}
\end{equation} 
\begin{equation}
 \sqrt[m]{\sqrt[n]{a}}=a^{\frac{1}{m\cdot n}}=\sqrt[m\cdot n]{a}
\end{equation} 
\begin{equation}
 \sqrt[n]{a}\cdot\sqrt[n]{b}=\sqrt[n]{a\cdot b}
\end{equation} 
\begin{equation}
 \frac{\sqrt[n]{a}}{\sqrt[n]{b}}=\sqrt[n]{\frac{a}{b}}
\end{equation} 
\end{subequations}
\end{shaded}

\subsection{Logarithmen}
Logarthmus ist das eindeutige lösen der Gleichung $r=a^x$ zur Lösung $x$.
\begin{shaded}
 \begin{equation}
  x=\log_{a}{r}
 \end{equation}
$a:\text{ Basis }(a>0,a\neq1)$ $r:\text{ Numerus }(r>0)$
\end{shaded}

\bla{Rechenregeln}
\begin{shaded}
\begin{subequations}
\begin{equation}
 \log_a b=\frac{\ln b}{\ln a}
\end{equation} 
\begin{equation}
 \log_a(u\cdot v) =\log_a u+\log_a v
\end{equation} 
\begin{equation}
 \log_a\left(\frac{u}{v}\right)=log_a u - \log_a v
\end{equation} 
\begin{equation}
 \log_a\left(u^k\right)=k\cdot\log_a u
\end{equation} 
\begin{equation}
 \log_a\sqrt[n]{u}=\left(\frac{1}{n}\right)\cdot\log_a u
\end{equation} 
\end{subequations}
\end{shaded}
\bla{Basiswechsel}
\begin{shaded}
 \begin{equation}
  \log_b r=\frac{\log_a r}{\log_a b}=\frac{1}{\log_a b}\cdot\log_a r =K\cdot\log_a r
 \end{equation}
\end{shaded}
Beim Basiswechsel von $a\rightarrow b$ werden die Logarithmen mit einer Konstanten K multipliziert.
\begin{shaded}
 \begin{align}
  \lg \rightarrow \ln \Rightarrow K&=2,3026 \\
  \ln \rightarrow \lg \Rightarrow K&=0,4343
 \end{align}
\end{shaded}

\subsection{Winkelfunktionen}
\begin{minipage}{.5\textwidth}
 \begin{shaded}\begin{align}
\sin{\alpha} &=\frac{a}{c}\\
\cos{\alpha} &=\frac{b}{c}\\
\tan{\alpha} &=\frac{a}{b}\\
\cot{\alpha} &=\frac{b}{a}
\end{align}
\end{shaded}
\end{minipage}\begin{minipage}{.5\textwidth}\begin{center} 
\begin{tikzpicture}[line width=.5mm]
%Dreieck
 \draw (0,0)--++(0:5cm)--++(90:3cm)--cycle;
\draw (2cm,0) arc (0:30.96:2cm);
\draw (4cm,0) arc (180:90:1cm);
%Deschriftung
\draw (4.6cm,.4)node{$\cdot$};
\draw (1.5cm,.4cm)node{$\alpha$};
\draw (5cm,1.5cm)node[right=1pt]{$a$};
\draw (2.5cm,0cm)node[below=1pt]{$b$};
\draw (30.96:2.9cm)node[above=1pt]{$c$};
\end{tikzpicture}
\end{center}
\end{minipage}

\bla{Rechenregeln}
\begin{shaded}
 \begin{align}
  \cos x &=\sin\left(x+\frac{\pi}{2}\right)&   \sin x &=\cos\left(x+\frac{\pi}{2}\right) \\
  \tan x &=\frac{\sin x}{\cos x}=\frac{1}{\cot x}& \cot x &=\frac{\cos x}{\sin x}=\frac{1}{\tan x}
\end{align}
\end{shaded}

\bla{Trigonometrischer Pythagoras}
\begin{shaded}
  \begin{equation}
\sin^2 x+ \cos^2 x =1
  \end{equation}
\end{shaded}

\bla{Addition von Winkeln}
\begin{shaded}
\begin{subequations}
  \begin{equation}
\sin\left(x_1\pm x_2\right)= \sin x_1 \cdot \cos x_2\pm \cos x_1 \cdot \sin x_2
  \end{equation}
  \begin{equation}
\cos\left(x_1\pm x_2\right)= \cos x_1 \cdot \cos x_2\mp \sin x_1 \cdot \sin x_2
  \end{equation}
  \begin{equation}
\tan\left(x_1\pm x_2\right)=\frac{\tan x_1 \pm \tan x_2}{1 \mp \tan x_1 \cdot \tan x_2}
  \end{equation}
  \begin{equation}
\cot\left(x_1\pm x_2\right)=\frac{\cot x_1 \cdot \cot x_2 \mp 1}{\cot x_2 \pm \cot x_1}
  \end{equation}
\end{subequations}
\end{shaded}

\bla{Multiplikation von Winkeln}
\begin{shaded}
\begin{subequations}
  \begin{equation}
\sin x_1 \cdot \sin x_2 =\frac{1}{2}\cdot \left(\cos (x_1 - x_2)- \cos(x_1+x_2)\right)
  \end{equation}
  \begin{equation}
\cos x_1 \cdot \cos x_2 = \frac{1}{2}\cdot \left(\cos(x_1 -x_2)+\cos(x_1+x_2)\right)
  \end{equation}
  \begin{equation}
\sin x_1 \cdot \cos x_2 =\frac{1}{2}\cdot \left(\sin(x_1 -x_2)+ \sin(x_1+x_2)\right)
  \end{equation}
  \begin{equation}
\tan x_1 \cdot \tan x_2 =\frac{\tan x_1 + \tan x_2}{\cot x_1+\cot x_2} 
  \end{equation}
\end{subequations}
\end{shaded}
\bla{Umrechnung Grad- $\Rightarrow$ Bogenmaß}
\begin{shaded}
 \begin{equation}
  x=\frac{\pi}{180^\circ}\cdot\alpha
 \end{equation}
\end{shaded}
\bla{Umrechnung Bogen- $\Rightarrow$ Gradmaß}
\begin{shaded}
 \begin{equation}
  \alpha=\frac{180^\circ}{\pi}\cdot x
 \end{equation}
\end{shaded}

Für weitere Winkelformeln siehe Papula Formelsammlung Seite 90-102.
\subsection{Fakultät}
$n!$ ist definitionsgemäß das Produkt aus denn ersten $n$ Faktoren
\begin{shaded}
 \begin{equation}
  n!=1\cdot 2\cdot 3 \ldots \left(n-1\right)\cdot n= \prod_{k=1}^{n}k \quad \left(n\in \mathbb{N}\right)
 \end{equation}
\end{shaded}
\bla{Vorsicht bei $0$ Fakultät}
\begin{shaded}
\begin{equation}
 0!=1
\end{equation}
\end{shaded}

\subsection{Binomischer Lehrsatz}
\begin{shaded}
 \begin{align}
  \left(a+b\right)^n&=a^n+\binom{n}{1}a^{n-1}\cdot b^1+\binom{n}{2}a^{n-2}\cdot b^2+\ldots+\binom{n}{n-1}a^{1}\cdot b^{n-1}+b^n \\
  &=\sum_{k=0}^{n}\binom{n}{k}a^{n-k}\cdot b^k \\
  &=\sum_{k=0}^{n}\binom{n}{k}a^{k}\cdot b^{n-k} 
 \end{align}
\end{shaded}

Der \emph{Binomialkoeffizienten} mit den Koeffizienten $\binom{n}{k}$ wird \emph{n über k} gelesen.

\bla{Bildungsgesetz}

\begin{shaded}
\begin{equation}
 \binom{n}{k}=\frac{n\cdot(n-1)\cdot(n-2)\cdot\ldots\cdot(n-(k-1))}{k!}=\frac{n!}{k!\cdot(n-k)!}
\end{equation}
\end{shaded}

\bla{Rechenregel}
\begin{shaded}
\begin{subequations}
\begin{equation}
\binom{n}{0}=\binom{n}{n}=1
\end{equation} 
\begin{equation}
\binom{n}{k}=0 \text{ für } k>n
\end{equation} 
\begin{equation}
\binom{n}{1}=\binom{n}{n-1}=n
\end{equation} 
\begin{equation}
\binom{n}{k}=\binom{n}{n-k}
\end{equation} 
\begin{equation}
\binom{n}{k}+\binom{n}{k+1}=\binom{n+1}{k+1}
\end{equation} 
\end{subequations}
\end{shaded}

\bla{Ersten Binomischen Formeln}

\begin{shaded}
 \begin{align}
  \left(a+b\right)^2 & =a^2+2\cdot a\cdot b+b^2\\
  \left(a+b\right)^3 & =a^3+3\cdot a^2\cdot b+3\cdot a\cdot b^2+b^3\\
  \left(a+b\right)^4 & =a^4+4\cdot a^3\cdot b+6\cdot a^2\cdot b^2+4\cdot a\cdot b^3+b^4
 \end{align}
\end{shaded}\begin{shaded}
 \begin{align}
  \left(a-b\right)^2 & =a^2-2\cdot a\cdot b+b^2\\
  \left(a-b\right)^3 & =a^3-3\cdot a^2\cdot b+3\cdot a\cdot b^2-b^3\\
  \left(a-b\right)^4 & =a^4-4\cdot a^3\cdot b+6\cdot a^2\cdot b^2-4\cdot a\cdot b^3+b^4 
\end{align}
\end{shaded}\begin{shaded}
 \begin{align}
\left(a+b\right)\cdot\left(a-b\right) & = a^2-b^2
 \end{align}
\end{shaded}

\subsection{Grenzwertberechnung}
\bla{Rechenregeln}
\begin{shaded}
\begin{subequations}
\begin{equation}
\lim_{x\to x_0}C\cdot f(x)= C\cdot \left( \lim_{x\to x_0}f(x)\right)
\end{equation} 
\begin{equation}
\lim_{x \to x_0}\left(f(x)\pm g(x)\right)= \lim_{x \to x_0} f(x)\pm \lim_{x \to x_0} g(x)
\end{equation} 
\begin{equation}
\lim_{x \to x_0} \left( f(x)\cdot g(x)\right) =\left(\lim_{x \to x_0}f(x)\right) \cdot \left(\lim_{x \to x_0}g(x)\right)
\end{equation} 
\begin{equation}
\lim_{x \to x_0}\frac{f(x)}{g(x)} =\frac{\lim_{x \to x_0}f(x)}{\lim_{x \to x_0}g(x)}
\end{equation} 
\begin{equation}
\lim_{x \to x_0}\sqrt[n]{f(x)}=\sqrt[n]{\lim_{x \to x_0}f(x)}
\end{equation} 
\begin{equation}
 \lim_{x \to x_0}\left(f(x)\right)^n=\left(\lim_{x \to x_0}f(x)\right)^n
\end{equation}
\begin{equation}
 \lim_{x \to x_0}\left(a^{f(x)}\right)=a^{\left(\lim_{x \to x_0}f(x)\right)}
\end{equation}
\begin{equation}
\lim_{x \to x_0}\left(\log_a f(x)\right)= \log_a \left(\lim_{x \to x_0}f(x)\right) 
\end{equation}
\end{subequations}
\end{shaded}


\bla{Grenzwertregel von Bernoulli und de l`Hospital}
Diese Regel wird angewendet wenn das normale Ergebniss die Form $\frac{0}{0}$ oder $\frac{\infty}{\infty}$ annimmt, was sonst eine beliebige Zahl darstellt.

\begin{shaded}
 \begin{equation}
  \lim_{x \to x_0}\frac{f(x)}{g(x)}=\lim_{x \to x_0}\frac{f'(x)}{g'(x)}
 \end{equation}
\end{shaded}

\bla{Berechnete Grenzwerte}
\begin{shaded}
 \begin{align}
\lim_{x \to \infty}\frac{1}{x}&=0 & \lim_{x \to \infty}a^x&=0 \text{ für } |a|<0 \\
\lim_{x \to \infty}\frac{a^x}{x!}&=0 & \lim_{x \to \infty}a^x&=1 \text{ für } a=1 \\
\lim_{x \to \infty}\sqrt{x}{a}&=1\text{ für } a>0& \lim_{x \to \infty}\frac{\sin x}{x}&=1\\
\lim_{x \to \infty}\left(1+\frac{1}{x}\right)^x&=\mathrm{e}
\end{align}
\end{shaded}

\subsection{Reihen}
\bla{Arithmetische Reihen}
\begin{shaded}
 \begin{equation}
  a+(a+d)+(a+2\cdot d)+\ldots+(a+(n-1)\cdot d)=\frac{n}{2}\left(2\cdot a+(n-1)\cdot d\right)
 \end{equation}
\end{shaded}
$a:$ Anfangsglied\quad $a_n=a+(n-1)\cdot d:$ Endglied
\bla{Geometrische Reihen}
\begin{shaded}
 \begin{equation}
  a+a\cdot q +a\cdot q^2+\ldots+a\cdot q^{n-1}=\sum_{k=1}^n a\cdot q^{k-1}=\frac{a(q^n-1)}{q-1}
 \end{equation}
\end{shaded}
$a:$ Anfangsglied\quad $a_n=a\cdot q^{n-1}:$ Endglied

\subsection{Koordinatensystem}

\begin{minipage}{.5\textwidth}
\bla{Kartesische Koordinaten}
 \begin{shaded}
  $0:$ Ursprung, Nullpunkt \\
  $x:$ Abzisse \\ 
  $y:$ Ordinate
 \end{shaded}

\bla{Polar Koordinaten}
 \begin{shaded}
  $0:$ Pol \\
  $r:$ Abstand des Punktes P zum Punkt O \\ 
  $\varphi:$ Winkel zwischen dem Strahl und der x Achse(\emph{Polarachse}) 
 \end{shaded}
\end{minipage}\begin{minipage}{.5\textwidth}\begin{center} 
 \begin{tikzpicture}
%Diagramm
  \draw [->] (0,0) -> (0,3cm)node[left=1pt]{$y$};
  \draw [->] (0,0)--(5cm,0)node[below=1pt]{$x$};
  \draw (0,0) node[left=1pt]{0};
%Koordinatensystem Punkt
  \draw ++(0,2.5cm)--++(2.5,0)--+(0,-2.5);
  \filldraw (2.5,2.5) circle (1pt);
  \draw (2.5,2.5cm)node[above=1pt]{$P=(x;y)$};
%Klammern
\begin{scope}[decoration={brace,amplitude=1.5mm}]
   \draw[decorate] (2.5,0) --(0,0);
   \draw[decorate] (2.5,2.5) --(2.5,0);
\end{scope}
%Beschriftung der Klammern
\draw (1.25,0)node[below=1pt]{$\Delta x$};
\draw (2.5,1.25)node[right=1pt]{$\Delta y$};
%Polarzeug
\draw (0,0)--(2.5,2.5);
\draw (1cm,0) arc (0:45:1cm);
\draw (45:1.77cm)node[above=1pt]{$r$};
\draw (22.5:.7cm)node{$\varphi$};
 \end{tikzpicture}\end{center}
\end{minipage}
\vspace{4pt}

\bla{Polarkoordinaten  $\Rightarrow$ Kartesische Koordinaten}

\begin{shaded}
 \begin{align}
  x&=r\cdot \cos \varphi& y=&r\cdot\sin\varphi
 \end{align}
\end{shaded}


\bla{Kartesische Koordinaten $\Rightarrow$ Polarkoordinaten}

\begin{shaded}
 \begin{align}
  r&=\sqrt{x^2+y^2}& \varphi=&\tan^{-1}\left(\frac{y}{x}\right)
 \end{align}
\end{shaded}

\bla{Koordinatentransformation(Parallelverschiebung)}

\begin{shaded}
 \begin{equation}
    y=f(x)\Rightarrow\left.\begin{aligned}
                            x&=u+a\\
			    y&=v+b
                           \end{aligned}\right.\Rightarrow v=f(u+a)-b
 \end{equation}
\end{shaded}
$(a;b)$: Ursprung des neuen u,v Koordinatensystems, bezogen auf das alte x,y-System. 

\section{Gleichungen}
\subsection{Gleichungen \emph{n}-ten Grades}
\begin{shaded}
 \begin{equation}
  a_n\cdot x^n+a_{n-1}\cdot x^{n-1}+\ldots+a_1\cdot x+a_0=0\quad (a_n\neq0,a_k\in\mathbb{R})
 \end{equation}

\end{shaded}
\bla{Eigenschaften}
\begin{itemize}
 \item Die Gleichung besitzen maximal $n$ reelle Lösungen.
\item Es gibt genau $n$ komplexe Lösungen.
\item Für ungerades $n$ gibt es mindestens eine reelle Lösung.
\item Komplexe Lösungen treten immer Paarweise auf.
\item Es existieren nur Lösungsformeln bis $n\leq 4$. Für $n>4$ gibt es nur noch grafische oder numerische Lösungswege.
\item Wenn eine Nullstelle bekannt ist kann man die Gleichung um einen Grad verringern, indem man denn zugehörigen Linearfaktor $x -x_1$ abspaltet(Polynome Division).
\end{itemize}

\subsection{Lineare Gleichungen}
\begin{shaded}
 \begin{equation}
  a_1\cdot x+a_0=0 \Rightarrow x_1=-\frac{a_0}{a_1}\quad (a_1\neq 0)
 \end{equation}
\end{shaded}

\subsection{Quadratische Gleichungen}
\begin{shaded}
 \begin{equation}
  a_2\cdot x^2+a_1\cdot x+a_0=0\quad (a_2\neq0)
 \end{equation}
\end{shaded}

\bla{Normalform mit Lösung}
\begin{shaded}
 \begin{equation}
  x^2+p\cdot x+q=0\Rightarrow x_{1/2}=-\frac{p}{2}\pm\sqrt{\left(\frac{p}{2}\right)^2-q}
 \end{equation}
\end{shaded}

\bla{Überprüfung (Vietascher Wurzelsatz)}
\begin{shaded}
 \begin{align}
  x_1+x_2&=-p& x_1\cdot x_2&=q 
 \end{align}
\end{shaded}
$x_1, x_2:$ Lösung der quadratischen Gleichung.

\subsection{Biquadratische Gleichungen}
Diese Gleichungen lassen sich mithilfe der Substitution lösen.
\begin{shaded}
 \begin{align}
  a\cdot x^4+b\cdot x^2 +c&=0&u&=x^2 \\
  a\cdot u^2+b\cdot u +c&=0&x&=\pm\sqrt{u}
 \end{align}
\end{shaded}
Das $u$ kann mithilfe der Lösungsformel einer quadratischen Gleichung gelöst werden.

\subsection{Gleichungen höheren Grades} 
Gleichungen höheren Grades kann man durch graphische oder numerische Ansätze lösen. Hilfreich ist das finden einer Lösung und das abspalten eines Linearfaktor
, mithilfe der Polynomendivision oder dem Hornor Schema,von der ursprünglichen Gleichung.

\bla{Polynomendivision}
\begin{shaded}
 \begin{equation}
  \frac{f(x)}{x-x_0}=\frac{a_3\cdot x^3+a_2\cdot x^2+a_1\cdot x +a_0}{x-x_0}=b_2\cdot x^2+b_1 \cdot x+b_0+r(x)
 \end{equation}
\end{shaded}
$x_0$ ist dabei die erste gefunden Nullstelle. r(x) verschwindet wenn $x_0$ ein Nullstellen oder eine Lösung von f(x) ist.
\begin{shaded}
 \begin{equation}
  r(x)=\frac{a_3\cdot x_0^3+a_2\cdot x_0^2+a_1\cdot x_0 +a_0}{x-x_0}=\frac{f(x_0)}{x-x_0}
 \end{equation}
\end{shaded}

\bla{Hornor-Schema}

\noindent\begin{tabularx}{\linewidth}{l|lllX}
\toprule
& $a_3$ & $a_2$ & $a_1$ & $a_0$\\
\midrule
\heboxc{$x_0$} & & $a_3\cdot x_0$ & $(a_2+a_3\cdot x_0)\cdot x_0$ &$(a_1+a_2\cdot x_0 +a_3\cdot x_0^2)\cdot x_0$ \\
& \heboxc{$a_3$} & $a_2+a_3\cdot x_0$ & $a_1+a_2\cdot x_0 + a_3\cdot x_0^2$ & $a_0+a_1\cdot x_0+a_2\cdot x_0^2+a_3\cdot x_0^3$\\
\hline
& $b_2$ & $b_1$ & $b_0$& $f(x_0)$ \\
\bottomrule
\end{tabularx}
\subsection{Wurzelgleichung}
Wurzelgleichungen löst man durch quadrieren oder mit hilfe von Substitution.
Bei Wurzelgleichung ist zu beachten das quadrieren keine Aquivalente Umformung ist und das 
Ergebniss überprüft werden muss.

\subsection{Ungleichungen}
\begin{itemize*}
\item Beidseitiges Subtrahieren oder Addieren ist möglich
\item Die Ungleichung darf mit einer beliebige positiven Zahl multipliziert oder dividiert werden
\item Die Ungleichung darf mit einer beliebige negativen Zahl multipliziert oder dividiert werden, wenn man gleichzeitig das Relationszeichen umdreht.
\end{itemize*}

\subsection{Betragsgleichungen}
Betragsgleichungen löst man mithilfe der Fallunterscheidung. Dabei wird einmal davon ausgegangen das der Term inerhalb des Betrags einmal positiv und einmal negativen
sein kann.   
\begin{shaded}
\begin{equation}
y=|x|=\left\{
\begin{aligned}
  &x \\
- &x 
\end{aligned}
\right. \text{für} \left. \begin{aligned}
x &\geq 0 \\
x &< 0 
\end{aligned} \right\} 
\end{equation}
\end{shaded}

\subsection{Interpolationspolynome}
Entwicklung einer Polynomefunktion anhand von $n+1$ Kurvenpunkten.
\begin{description*}
 \item[1. Möglichkeit] Aufstellen von $n+1$ Gleichungen und ermitteln der Kurvenfunktion mithilfe des Gaußen Algorithmus.
 \item[2. Möglichkeit] Interpolationspolynome von Newton
\end{description*}

\bla{Interpolationspolynome von Newton}

\begin{merkbox}Gegeben sind die Punkte $P_0=(x_0;y_0)$, $P_1=(x_1;y_1)$, $P_2=(x_2;y_2)$, $\ldots$, $P_n=(x_n;y_n)$, damit lautet die Funktion wie folgt:
\begin{align}
 f(x)=a_0&+a_1\cdot (x-x_0)+ a_2\cdot (x-x_0)\cdot(x-x_1)\\
	 &+a_3\cdot(x-x_0)\cdot(x-x_1)\cdot(x-x_2)\\
	 &+\ldots\\
	 &+a_n\cdot(x-x_0)\cdot\ldots\cdot\cdot(x-x_{n-1})
\end{align}
Die Koeffizienten$a_0, a_1, a_2,\ldots, a_n$ lassen sich mithilfe des Differentenshema berechnen. Dabei ist $y_0=a_0$, $[x_0,x_1]=a_1$, $[x_0,x_1,x_2]=a_2$
 usw. 
\end{merkbox}

\bla{Differentenshema}

\noindent\begin{tabularx}{\linewidth}{cccccccX}
\toprule
 k	&$x_k$	&$y_k$		&$1$		&$2$		&$3$			&$\ldots$	\\ \midrule
 $0$	&$x_0$	&\hebox{$y_0$}	&		&		&			&		\\
	&	&		&\hebox{$[x_0,x_1]$}	&		&			&		\\
 $1$	&$x_1$	&$y_1$		&		&\hebox{$[x_0,x_1,x_2]$}&			&		\\
 	&	&		&$[x_1,x_2]$	&		&\hebox{$[x_0,x_1,x_2,x_3]$}	&		\\
 $2$	&$x_2$	&$y_2$		&		&\heboxc{$[x_1,x_2,x_3]$}&			&$\ldots$	\\
 	&	&		&$[x_2,x_3]$	&		&\heboxc{$[x_1,x_2,x_3,x_4]$}	&		\\
 $3$	&$x_3$	&$y_3$		&		&\heboxc{$[x_2,x_3,x_4]$}&			&$\ldots$	\\
 	&	&		&$\ldots$	&		&$\ldots$		&		\\
 $\vdots$&$\vdots$&$\vdots$	&		&		&			&		\\
 $n$	&$x_n$	&$y_n$		&		&		&			&		\\ \bottomrule
\end{tabularx}

\bla{Rechenregel für dividierte Differenzen}
\begin{shaded}
\begin{minipage}{.5\textwidth}

 \begin{equation}
 \left.\begin{aligned}
  [x_0,x_1]&=\frac{y_0-y_1}{x_0-x_1}\\
  [x_1,x_2]&=\frac{y_1-y_2}{x_1-x_2} \\
\vdots &
 \end{aligned}\right.
\end{equation} 
\end{minipage}\begin{minipage}{.5\textwidth}
 \begin{equation}
 \left.\begin{aligned}
  [x_0,x_1,x_2]&=\frac{[x_0,x_1]-[x_1,x_2]}{x_0-x_2}\\
  [x_1,x_2,x_3]&=\frac{[x_1,x_2]-[x_2,x_3]}{x_1-x_3} \\
\vdots &
 \end{aligned}\right.
\end{equation} 

\end{minipage}\end{shaded}
\begin{shaded}
 \begin{equation}
 \left.\begin{aligned}
  [x_0,x_1,x_2,x_3]&=\frac{[x_0,x_1,x_2]-[x_1,x_2,x_3]}{x_0-x_2}\\
  [x_1,x_2,x_3,x_4]&=\frac{[x_1,x_2,x_3]-[x_2,x_3,x_4]}{x_1-x_3} \\
\vdots &
 \end{aligned}\right.
\end{equation} 
\end{shaded}

\section{Differntialrechnung}

\begin{boxleft}
  Potenzfunktion
\end{boxleft}\begin{boxrightshaded}
 \begin{align}
  &x^n& 	&n\cdot x^{n-1}
 \end{align}
\end{boxrightshaded}


\rule{\textwidth}{.2pt}

\begin{boxleft}
  Exponentialfunktionen
\end{boxleft}\begin{boxrightshaded} \begin{align}
  &e^x& 	&e^x\\
  &a^x& 	&\ln a\cdot a^x
 \end{align}
\end{boxrightshaded}

\boxrule

\begin{boxleft}
  Logarithmusfunktionen
\end{boxleft}\begin{boxrightshaded}
 \begin{align} 
  &\ln x& &\frac{1}{x}\\
  &\log_a x&	&\frac{1}{(\ln a)\cdot x}
 \end{align}
\end{boxrightshaded}

\boxrule

\begin{boxleft}
  Trigonometrische Funktionen
\end{boxleft}\begin{boxrightshaded}
 \begin{align} 
  &\sin x& 	&\cos x \\
  &\cos x& 	&-\sin x\\
  &\tan x&	&\frac{1}{\cos^2 x}\\
  &\tan x&	&1+\tan^2 x
 \end{align}\end{boxrightshaded}
            
\boxrule

\begin{boxleft}
  Faktorregel
\end{boxleft}\begin{boxrightshaded}
 \begin{align} 
&y=C\cdot f(x)& \Rightarrow &y'=C\cdot f'(x)
 \end{align}\end{boxrightshaded}
            
%\chapter{Physik}
\section{Kinematik}
\subsection{Geradlinige Bewegungen}
\bla{Gesetze}
\begin{shaded}
\begin{align}
	a(t)&=a_0=\frac{\diff v}{\diff t}=\dot{v}=\ddot{s} \\
	v(t)&=a_0*t+v_0=\frac{\diff s}{\diff t}=\dot{s} \\
	s(t)&=\frac{1}{2}a_0*t^2+v_0*t+s_0
\end{align}
\end{shaded}
\subsection{Kreisbewegungen}
 \bla{Gesetze}

\begin{boxleft}
Winkelgrößen
\end{boxleft}\begin{boxrightshaded}
\begin{align}
\alpha(t)&=\alpha_0=\frac{\diff \omega}{\diff t}=\dot{\omega}=\ddot{\varphi} \\
\omega(t)&=\alpha_0*t+\omega_0=\frac{\diff \varphi}{\diff t}=\dot{\varphi} \\
\varphi(t)&=\frac{1}{2}\alpha_0*t^2+\omega_0*t+\varphi_0
\end{align}
\end{boxrightshaded}

\begin{boxleft}
Bahngrößen
\end{boxleft}\begin{boxrightshaded}
\begin{align}
a_t(t)&=a_0=\frac{\diff v}{\diff t}=\dot{v}=\ddot{s} \\
v(t)&=a_0*t+v_0=\frac{\diff s}{\diff t}=\dot{s} \\
s(t)&=\frac{1}{2}a_0*t^2+v_0*t+s_0
\end{align}
\end{boxrightshaded}

\begin{boxleft}Winkelgeschwindigkeit,\\
Kreisfrequenz
\end{boxleft}\begin{boxrightshaded}
\begin{align}
\omega&=\frac{2\cdot\pi}{T}\\
&=2\cdot\pi\cdot n \\
&=2\cdot\pi\cdot f
\end{align}
\end{boxrightshaded}

\begin{boxleft}Bahngeschwindigkeit
\end{boxleft}\begin{boxrightshaded}
\begin{align}
v&=\frac{2\cdot \pi \cdot r}{T}\\
&=\omega\cdot r
\end{align}
\end{boxrightshaded}

\begin{boxleft}Radialbeschleunigung
\end{boxleft}\begin{boxrightshaded}
\begin{align}
a_r&=\frac{v^2}{r}\\
&=v\cdot\omega\\
&=\omega^2\cdot r
\end{align}
\end{boxrightshaded}

\begin{boxleft}Umdrehungen
\end{boxleft}\begin{boxrightshaded}
\begin{align}
N	&=\frac{\omega_0\cdot t}{2\cdot \pi}+\frac{1}{2}\cdot\frac{\alpha}{2\cdot \pi}\cdot t^2\\
	&=n_0\cdot t+\frac{\alpha}{4\cdot\pi}\cdot t^2
\end{align}
\end{boxrightshaded}

\begin{boxleft}Umrechnung\\
Winkelgrößen $\Leftrightarrow$ Bahngrößen
\end{boxleft}\begin{boxrightshaded}
\begin{align}
a_t		&=\alpha\cdot r\\
\vec{a_t}	&=\vec{\alpha} \times \vec{r}\\
v		&=\omega\cdot r\\
\vec{v}		&=\vec{\omega}\times\vec{r}\\
s		&=\varphi\cdot r
\end{align}
\end{boxrightshaded}
%\chapter{Elektrotechnik}
\section{Grundgrößen}

\begin{boxleft}Elementarladung
\end{boxleft}\begin{boxrightshaded}
\begin{align}
e\approx 1,6\cdot 10^{-19}C
\end{align}
\end{boxrightshaded}

\begin{boxleft}ele. Ladung
\end{boxleft}\begin{boxrightshaded}
\begin{align}
\left[Q\right]&=1C=1As\\
Q&=n\cdot e
\end{align}
\end{boxrightshaded}

\begin{boxleft}ele. Strom
\end{boxleft}\begin{boxrightshaded}
\begin{align}
\left[I\right]&=1A\\
i(t)&=\frac{\diff Q}{\diff t}
\end{align}
\end{boxrightshaded}

\begin{boxleft}ele. Potenzial
\end{boxleft}\begin{boxrightshaded}
\begin{align}
\left[\varphi\right]&=1V=1\frac{Nm}{As}=1\frac{kgm^2}{As^3}\\
\varphi&=\frac{W}{Q}
\end{align}
\end{boxrightshaded}

\begin{boxleft}ele. Spannung
\end{boxleft}\begin{boxrightshaded}
\begin{align}
\left[U\right]&=1V\\
U_{AB}&=\varphi_a-\varphi_b
\end{align}
\end{boxrightshaded}
%\chapter{Signal- und Systemtheorie}

\section{Grundsignale}

\subsection{Einheitsignale}
\begin{boxleft}\bla{Diracstoß}
\des[\per\second]{\delta\left(t\right)}{Diracstoß}
\end{boxleft}\begin{boxrightshaded}
\begin{align*}
\delta\left(t\right)&=
\begin{dcases*}
  0\si{\per\second}&$t<0$\\
\infty\si{\per\second}& $t=0$\\
  0\si{\per\second}&$t>0$
\end{dcases*}\\
\int_{-\infty}^\infty\delta\left(t\right)\diff t&=1\\
\delta\left(t\right)&=\frac{\diff \sigma\left(t\right)}{\diff t}=\frac{\diff^2 \alpha\left(t\right)}{\diff t^2}
\end{align*}
\end{boxrightshaded}


\begin{boxleft}\bla{Einheitssprungsfunktion}
\des{\sigma\left(t\right)}{Einheitssprungsfunktion}
\end{boxleft}\begin{boxrightshaded}
\begin{align*}
\sigma\left(t\right)&=
\begin{dcases*}
  0&$t<0$\\
0,5& $t=0$\\
  1&$t>0$
\end{dcases*}\\
\sigma\left(t\right)&=\int_{-\infty}^t \delta\left(t\right)\diff t=\frac{\diff \alpha\left(t\right)}{\diff t}
\end{align*}
\end{boxrightshaded}


\begin{boxleft}\bla{Einheitsanstiegsfunktion}
\des[\second]{\alpha\left(t\right)}{Einheitsanstiegsfunktion}
\end{boxleft}\begin{boxrightshaded}
\begin{align*}
\alpha\left(t\right)&=
\begin{dcases*}
  0\si{\second}&$t<0$\\
t& $t=0$
\end{dcases*}\\
\alpha\left(t\right)&=\iint_{-\infty}^t \delta\left(t\right)\diff t\diff t=\int_{-\infty}^t \sigma\left(t\right)\diff t
\end{align*}
\end{boxrightshaded}


\subsection{Weitere Grundsignale}
\begin{boxleft}\bla{Rechtecksimpuls}
\des{\operatorname{rect}_T\left(t\right)}{Rechtecksimpuls}
\end{boxleft}\begin{boxrightshaded}
\begin{align*}
\operatorname{rect}_T\left(t\right)&=
\begin{dcases*}
  1&$\left|t\right|<\frac{T}{2}$\\
0,5& $\left|t\right|=\frac{T}{2}$\\
0& $\left|t\right|>\frac{T}{2}$
\end{dcases*}
\end{align*}
\end{boxrightshaded}


\begin{boxleft}\bla{Dreiecksimpuls}
\des{\operatorname{\Lambda}_T\left(t\right)}{Dreiecksimpuls}
\end{boxleft}\begin{boxrightshaded}
\begin{align*}
\operatorname{\Lambda}_T\left(t\right)&=
\begin{dcases*}
  1+\frac{t}{T}&$-T<t<0$\\
  1-\frac{t}{T}&$0\leq t<T$\\
  0&$\left|t\right|>T$\\
\end{dcases*}
\end{align*}
\end{boxrightshaded}

\subsection{Signalveränderungen}
\begin{boxleft}\bla{Offset}
\des{X_{off}}{Offsetwert}
\end{boxleft}\begin{boxrightshaded}
\begin{align*}
x_2\left(t\right)&=x_1\left(t\right)+X_{off}
\end{align*}
\end{boxrightshaded}


\begin{boxleft}\bla{Skalierung}
\des{V}{Verstärkungsfaktor}
\end{boxleft}\begin{boxrightshaded}
\begin{align*}
x_2\left(t\right)&=V\cdot x_1\left(t\right)
\end{align*}
\end{boxrightshaded}


\begin{boxleft}\bla{Verschiebung}
\des{t_0}{Verschiebungskonstante}
\end{boxleft}\begin{boxrightshaded}
\begin{align*}
x_2\left(t\right)&= x_1\left(t-t_0\right)&&\text{$t_0>0$: Rechtsverschiebung}
\end{align*}
\end{boxrightshaded}

\begin{boxleft}\bla{Negation des Argumentes}
\end{boxleft}\begin{boxrightshaded}
\begin{align*}
x_2\left(t\right)&= x_1\left(-t\right)&&\text{Spiegelung an der Ordinate}
\end{align*}
\end{boxrightshaded}


\begin{boxleft}\bla{Negiertes und verschobenes Argument}
\des{t_0}{Verschiebungskonstante}
\end{boxleft}\begin{boxrightshaded}
\begin{align*}
x_2\left(t\right)&= x_1\left(-\left(t-t_0\right)\right)&&\text{Spiegelung bei $\frac{t_0}{2}$}
\end{align*}
\end{boxrightshaded}


\begin{boxleft}\bla{Argumentskalierung}
\end{boxleft}\begin{boxrightshaded}
\begin{align*}
x_2\left(t\right)&= x_1\left(a\cdot t\right)&&\text{$a<1$ Streckung der Funktion}
\end{align*}
\end{boxrightshaded}



\section{Signaleigenschaften}

\subsection{Energiesignale}

\begin{boxshaded}
\begin{align*}
&\text{$E=$endlich positiver Wert. $P=0$} 
\end{align*}
\end{boxshaded}

\begin{boxleft}\bla{Energie}
\des[\watt\second]{E_R}{Energie}\\
\des{E_X}{Normierte Signalenergie}
\end{boxleft}\begin{boxrightshaded}
\begin{align*}
E_R&=\int_{-\infty}^\infty u\left(t\right)\cdot i\left(t\right)\diff t\\
&=\int_{-\infty}^\infty \frac{u^2\left(t\right)}{R}\diff t\\
E_x&=m_{i2}=\int_{-\infty}^\infty x^2\left(t\right)\diff t&&\text{Normierung auf $R=1$}\\
E_x&=\sum_{k=-\infty}^{\infty}x^2\left(k\right)
\end{align*}
\end{boxrightshaded}

\begin{boxleft}\bla{Impulsfläsche}
\des[\watt\second]{E_R}{Energie}
\end{boxleft}\begin{boxrightshaded}
\begin{align*}
A_x&=m_{i1}=\int_{-\infty}^\infty x\left(t\right)\diff t\\
A_x&=\sum_{k=-\infty}^{\infty}x\left(k\right)
\end{align*}
\end{boxrightshaded}


\subsection{Leistungssignale}


\begin{boxshaded}
\begin{align*}
&\text{$E=\infty$. $P=$endlich positiver Wert.} 
\end{align*}
\end{boxshaded}

\begin{boxleft}\bla{Mittlere Signalleistung}
\des{P_x}{Mittlere Signalleistung}\\
\des{\bar{x^2}}{quadratischer Mittelwert}\\
\des{m_2}{gewöhnliches Moment 2. Ordnung}\\
\des{x_0^2}{Konstantes Signale}
\end{boxleft}\begin{boxrightshaded}
\begin{align*}
P_x&=\bar{x^2}=m_2=\lim_{T\to \infty}\int_{t_0}^{t_0+T} x^2\left(t\right)\diff t\\
&=\frac{1}{n\cdot T_p}\int_{t_0}^{t_0+n\cdot T_p} x^2\left(t\right)\diff t&&\text{Periodische Signale}\\
&=\lim_{N\to \infty}\frac{1}{N}\sum_{k=k_0}^{k_0+N-1}x^2\left(k\right)\\
&=\frac{1}{N_p}\sum_{k=k_0}^{k_0+N_p-1}x^2\left(k\right)&&\text{Periodische Signale}\\
&=X_0^2&&\text{Konstantes Signale}
\end{align*}
\end{boxrightshaded}

\begin{boxleft}\bla{Effektivwert}
\des{x_{eff}}{Effektivwert}
\end{boxleft}\begin{boxrightshaded}
\begin{align*}
x_{eff}&=\sqrt{P_x}=\sqrt{\lim_{T\to \infty}\int_{t_0}^{t_0+T} x^2\left(t\right)\diff t}\\
&=\sqrt{\frac{1}{n\cdot T_p}\int_{t_0}^{t_0+n\cdot T_p} x^2\left(t\right)\diff t}&&\text{Periodische Signale}\\
&=\sqrt{\lim_{N\to \infty}\frac{1}{N}\sum_{k=k_0}^{k_0+N-1}x^2\left(k\right)}\\
&=\sqrt{\frac{1}{N_p}\sum_{k=k_0}^{k_0+N_p-1}x^2\left(k\right)}&&\text{Periodische Signale}\\
&=X_0&&\text{Konstantes Signale}
\end{align*}
\end{boxrightshaded}


\begin{boxleft}\bla{Gleichanteil}
\des{\bar{x}}{Gleichanteil}\\
\des{m_1}{gewöhnliches Moment 1. Ordnung}\\
\des{E\left(x\right)}{Erwartungswert}
\end{boxleft}\begin{boxrightshaded}
\begin{align*}
\bar{x}&=m_1=E\left(x\right)=\lim_{T \to \infty}\int_{t_0}^{t_0+T}x\left(t\right) \diff t\\
&=\frac{1}{n\cdot T_p}\int_{t_0}^{t_0+n\cdot T_p}u\left(t\right)\diff t&&\text{Periodische Signale}\\
&=\lim_{N\to\infty}\frac{1}{N}\sum_{k=k_0}^{k_0+N-1}x\left(t\right)\\
&=\frac{1}{N_p}\sum_{k=k_0}^{k_0+N_p-1}x\left(t\right)&&\text{Periodische Signale}\\
&=X_0&&\text{Konstantes Signale}
\end{align*}
\end{boxrightshaded}


\begin{boxleft}\bla{Signalgleichleistung}
\des{P_{x=}}{Signalgleichleistung}\\
\des{\bar{x}^2}{Quadratisch linearer Mittelwert}\\
\des{m_1^2}{Quadratisch g. Moment 1. Ordnung}
\end{boxleft}\begin{boxrightshaded}
\begin{align*}
P_{x=}&=\left(\bar{x}\right)^2=m_1^2=\left(\lim_{T \to \infty}\int_{t_0}^{t_0+T}x\left(t\right) \diff t\right)^2\\
&=\left(\frac{1}{n\cdot T_p}\int_{t_0}^{t_0+n\cdot T_p}u\left(t\right)\diff t\right)^2&&\text{Periodische Signale}\\
&=\left(\lim_{N\to\infty}\frac{1}{N}\sum_{k=k_0}^{k_0+N-1}x\left(t\right)\right)^2\\
&=\left(\frac{1}{N_p}\sum_{k=k_0}^{k_0+N_p-1}x\left(t\right)\right)^2&&\text{Periodische Signale}\\
&=\left(X_0\right)^2&&\text{Konstantes Signale}
\end{align*}
\end{boxrightshaded}


\begin{boxleft}\bla{Signalwechselleistung}
\des{P_{x~}}{Signalwechselleistung}\\
\des{\sigma^2}{Varianz}\\
\des{\mu_2}{Quadratisch g. Moment 2. Ordnung}
\end{boxleft}\begin{boxrightshaded}
\begin{align*}
P_{x~}&=\sigma^2=\mu_2=\bar{x^2}-\bar{x}^2\\
&=\lim_{T\to\infty}\frac{1}{T}\int_{t_0}^{t_0+T}\left(x\left(t\right)-\bar{x}\right)^2\diff t\\
&=\frac{1}{n\cdot T_p}\int_{t_0}^{t_0+n\cdot T_p}\left(x\left(t\right)-\bar{x}\right)^2\diff t&&\text{Periodische Signale}\\
&=\lim_{N\to\infty}\frac{1}{N}\sum_{k=k_0}^{k_0+N-1}\left(x\left(k\right)-\bar{x}\right)^2\\
&=\frac{1}{N_p}\sum_{k=k_0}^{k_0+N_p-1}\left(x\left(k\right)-\bar{x}\right)^2&&\text{Periodische Signale}\\
&=0&&\text{Konstantes Signale}
\end{align*}
\end{boxrightshaded}

\begin{boxleft}\bla{Standartabweichung}
\des{\sigma}{Standartabweichung}
\end{boxleft}\begin{boxrightshaded}
\begin{align*}
\sigma=\sqrt{P_{x~}}&=\sqrt{\mu_2}
\end{align*}
\end{boxrightshaded}


\section{Systeme}

\subsection{Linearität}

\begin{boxleft}\bla{Homogenität}
\end{boxleft}\begin{boxrightshaded}
\begin{align*}
x\left(t\right)&=C\cdot x_1\left(t\right)\to y\left(t\right)=C\cdot y_1\left(t\right)
\end{align*}
\end{boxrightshaded}

\begin{boxleft}\bla{Additivität}
\end{boxleft}\begin{boxrightshaded}
\begin{align*}
x\left(t\right)&= x_1\left(t\right)+x_2\left(t\right)\to y\left(t\right)= y_1\left(t\right)+y_2\left(t\right)
\end{align*}
\end{boxrightshaded}


\subsection{Zeitinvarianz}

\begin{boxleft}\bla{Zeitinvarianz}
\end{boxleft}\begin{boxrightshaded}
\begin{align*}
x\left(t\right)&=x_1\left(t-\tau\right)\to y\left(t\right)=y_1\left(t-\tau\right)
\end{align*}
\end{boxrightshaded}


\subsection{Kausalität}

\begin{boxleft}\bla{Zeitinvarianz}
\end{boxleft}\begin{boxrightshaded}
\begin{align*}
x\left(t\right)&=x_1\left(t-\tau\right)\to y\left(t\right)=y_1\left(t-\tau\right)
\end{align*}
\end{boxrightshaded}

% 
\chapter{Binäre Rechenoperation}

\section{Zahlensystem}

\begin{boxleft}\bla{Dualsystem}
\end{boxleft}\begin{boxrightshaded}
\begin{align*}
&\text{Basis}:2\quad z\in\left(1;0\right)\\
&\text{Format}:2^{n-1} \dots 2^{0},2^{-1} \dots 2^{-m}\\
&\text{Zahlenwert}:\sum_{l=-m}^{n-1}z_l\cdot2^l
\end{align*}
\end{boxrightshaded}

\begin{boxleft}\bla{Trenärsystem}
\end{boxleft}\begin{boxrightshaded}
\begin{align*}
\text{Basis}:2\quad z\in\left(1;0;-1\right)
\end{align*}
\end{boxrightshaded}

\begin{boxleft}\bla{Oktalsystem}
\end{boxleft}\begin{boxrightshaded}
\begin{align*}
\text{Basis}:8\quad z\in\left(0;1;2;3;4;5;6;7\right)
\end{align*}
\end{boxrightshaded}

\begin{boxleft}\bla{Hexadezimalsystem}
\end{boxleft}\begin{boxrightshaded}
\begin{align*}
\text{Basis}:16\quad z\in\left(0;1;2;3\dots d;e;f\right)
\end{align*}
\end{boxrightshaded}

\begin{boxleft}\bla{Dezimalsystem}
\end{boxleft}\begin{boxrightshaded}
\begin{align*}
\text{Basis}:10\quad z\in\left(0;1;2;3\dots 8;9\right)
\end{align*}
\end{boxrightshaded}

\begin{boxleft}\bla{Stellenberechnung}
\des{k}{Zahlenwert}\\
\des{l}{Anzahl Nachkommastellen in Dezimalsystem}\\
\des{n}{Anzahl Vorkommanstellen in Dualsystem}\\
\des{m}{Anzahl Nachkommastellen in Dualsystem}
\end{boxleft}\begin{boxrightshaded}
\begin{align*}
&n\geq\text{ceil}\left(\frac{\lg\left(k+1\right)}{\lg\left(2\right)}\right)\\
&m\geq\text{ceil}\left(\frac{l}{\lg\left( 2\right)}\right)
\end{align*}
\end{boxrightshaded}

\begin{boxleft}\bla{Wertebereich und Quantisierungsfehler(Dualsystem)}
\des{n}{Anzahl Vorkommanstellen in Dualsystem}\\
\des{m}{Anzahl Nachkommastellen in Dualsystem}
\end{boxleft}\begin{boxrightshaded}
\begin{align*}
&\text{Wertebereich}=0\dots 2^n-2^{-m}\\
&\text{Quantisierungsfehler}=\pm\frac{1}{2}\text{LSB}=\pm 2^{-m-1}\\
&\text{Quantisierungsstuffen}=2^{-m}
\end{align*}
\end{boxrightshaded}

\begin{boxleft}\bla{Umwandlung Negativer Dualzahlen}
\end{boxleft}\begin{boxrightshaded}
\begin{align*}
n&=\text{ceil}\left(1+\frac{\lg\left(|\text{Wert}|\right)}{\lg\left(2\right)}\right)\\
\Delta&=2^{n-1}-\left|\text{Wert}\right|\\
&\text{Umwadlung von $\Delta$ bis MSB Stelle -1, danach setzten der richtigen MSB-Stelle.}
\end{align*}
\end{boxrightshaded}

\section{Addition}

\subsection{Zwei Operanden}

\begin{boxleft}\bla{Halbaddierer}
\des{s_i}{Summe der Stelle i}\\
\des{c_{i+1}}{Übertrag der Stelle i+1}
\end{boxleft}\begin{boxrightshaded}
\begin{align*}
s_i&=a_i\nsim b_i\\
c_{i+1}&=a_i \cdot b_i
\end{align*}
\end{boxrightshaded}

\begin{boxleft}\bla{Volladdierer}
\des{s_i}{Summe der Stelle i}\\
\des{c_{i+1}}{Übertrag der Stelle i+1}\\
\des{c_{i}}{Übertrag der Stelle i}
\end{boxleft}\begin{boxrightshaded}
\begin{align*}
s_i&=a_i\nsim b_i \nsim c_i\\
c_{i+1}&=a_i \cdot b_i+\left(a_i\nsim b_i\right)\cdot c_i
\end{align*}
\end{boxrightshaded}

\begin{boxleft}\bla{Carry-Look-Ahead}
\des{s_i}{Summe der Stelle i}\\
\des{p_{i+1}}{Propagate Übertrag an der Stelle i}\\
\des{g_{i+1}}{Generate Übertrag kompensation}
\end{boxleft}\begin{boxrightshaded}
\begin{align*}
s_i&=a_i\nsim b_i \nsim c_i\\
p_{i+1}&=a_i\nsim b_i\\
g_{i+1}&=a_i\cdot b_i\\
c_{i+1}&=p_{i+1}\cdot c_i +g_{i+1}
\end{align*}
\end{boxrightshaded}

\subsection{Mehrere Operanden}

\begin{boxshaded}
\begin{align*}
&\text{Ripple-Carry: Jede Stuffe addiert jeweils ein Operand hinzu.}\\
&\text{Baumaddierer: Die einzelnen Operaden werden Baumförmig addiert.}\\
&\text{Ripple-Save: Nur der letzte Addierer ist Sequenziel aufgebaut. Daher der Übertrag des Vorgängers wird beim nächsten Aufaddiert.}
\end{align*}
\end{boxshaded}

\subsection{Überlauf}

\begin{boxleft}\bla{Positive Operanden}
\des{s}{Summe}\\
\des{n}{Anzahl Vorkommanstellen in Dualsystem}
\end{boxleft}\begin{boxrightshaded}
\begin{align*}
s&=\left(a+b\right)\text{mod} 2^n
\end{align*}
\end{boxrightshaded}

\begin{boxleft}\bla{Negative Operanden}
\des{s}{Entstehender Summen Wert}\\
\des{p}{Operanden Anzahl}
\end{boxleft}\begin{boxrightshaded}
\begin{align*}
s&=\left(a+b+2^n\text{ceil}\left(\frac{p-1}{2}\right)+2^{n-1}\right)\text{mod} 2^n-2^{n-1}
\end{align*}
\end{boxrightshaded}

\begin{boxleft}\bla{Vermeidung}
\des{r}{Zusätzliche Summenstellen}\\
\des{p}{Operanden Anzahl}
\end{boxleft}\begin{boxrightshaded}
\begin{align*}
r=\text{ceil}\left(\frac{\lg\left(p\right)}{\lg\left(2\right)}\right)
\end{align*}
\end{boxrightshaded}


\subsection{Überlaufserkennung}

\begin{boxleft}\bla{Vergleich MSB Stellen}
\end{boxleft}\begin{boxrightshaded}
\begin{align*}
&\begin{aligned}
&a_{MSB}\neq b_{MSB}\hfill&&\Rightarrow \text{Kein Überlauf möglich}\\
&a_{MSB}=b_{MSB}=s_{MSB}&&\Rightarrow \text{Kein Überlauf möglich}\\
&a_{MSB}=b_{MSB}=0 \quad\&\quad s_{MSB}=1&&\Rightarrow \text{Positiver Überlauf}\\
&a_{MSB}=b_{MSB}=1 \quad\&\quad s_{MSB}=0&&\Rightarrow \text{Negativer Überlauf}
\end{aligned}\\
&\mathit{MIN}=a_{MSB}\\
&\mathit{OVF}=\overline{a}_{MSB}s_{MSB}\left(\overline{b_{MSB}\nsim s_{MSB}}\right)+a_{MSB}\overline{s}_{MSB}\left(b_{MSB}\nsim s_{MSB}\right)
\end{align*}
\end{boxrightshaded}

\begin{boxleft}\bla{Vergleich des Carry}
\end{boxleft}\begin{boxrightshaded}
\begin{align*}
&c_{I;MSB}= c_{O;MSB}&&\Rightarrow \text{Kein Überlauf möglich}\\
&c_{I;MSB}=1\quad\&\quad c_{O;MSB}=0&&\Rightarrow \text{Positiver Überlauf}\\
&c_{I;MSB}=0\quad\&\quad c_{O;MSB}=1&&\Rightarrow \text{Negativer Überlauf}\\
&\mathit{MIN}=c_{O;MSB}\\
&\mathit{OVF}=c_{I;MSB}\nsim c_{O;MSB}
\end{align*}
\end{boxrightshaded}

\begin{boxleft}\bla{Erweiterung der MSB Stelle}
\end{boxleft}\begin{boxrightshaded}
\begin{align*}
&s_{MSB}= s_{MSB+1}&&\Rightarrow \text{Kein Überlauf möglich}\\
&s_{MSB}=1\quad\&\quad s_{MSB+1}=0&&\Rightarrow \text{Positiver Überlauf}\\
&s_{MSB}=0\quad\&\quad s_{MSB+1}=1&&\Rightarrow \text{Negativer Überlauf}\\
&\mathit{MIN}=s_{MSB}\\
&\mathit{OVF}=s_{MSB}\nsim s_{MSB+1}
\end{align*}
\end{boxrightshaded}

\begin{boxleft}\bla{Sättigung}
\des{s_{MSB}}{Behandlung der höchsten Stelle}\\
\des{s_{LSB}}{Behandlung der restlichen Stellen}
\end{boxleft}\begin{boxrightshaded}
\begin{align*}
s'_{MSB}&=s_{MSB}\overline{\mathit{OVF}}+\mathit{OVF}\mathit{MIN}\\
s'_{LSB}&=s_{LSB}\overline{\mathit{OVF}}+\mathit{OVF}\overline{{MIN}}
\end{align*}
\end{boxrightshaded}

\subsection{Umwandlung Tenärcode}

\begin{tabular}{lll|ll}
$h_i$& $b_{i+1}$&$b_i$&$c_i$&$h_{i+1}$\\
\hline
0&0&0&0&0\\
0&0&1&1&0\\
0&1&0&0&0\\
0&1&1&-1&1\\
1&0&0&1&0\\
1&0&1&0&1\\
1&1&0&-1&1\\
1&1&1&0&1
\end{tabular}
 
\chapter{Filterentwurf}

\section{Darstellung Übertragungsfunktion}

\begin{boxleft}\bla{Polynomedarstellung}
\end{boxleft}\begin{boxrightshaded}
\begin{align*}
G(p)&=\frac{\sum\limits_{i=-q}^rf_i\cdot p^i}{\sum\limits_{j=-k}^vf_j\cdot p^j}
\end{align*}
\end{boxrightshaded}

\begin{boxleft}\bla{Produktdarstellung}
\end{boxleft}\begin{boxrightshaded}
\begin{align*}
G(p)&=K\cdot\frac{\prod\limits_{i=1}^m\left(p-p_{oi}\right)}{\prod\limits_{j=1}^n\left(p-p_{pj}\right)}
\end{align*}
\end{boxrightshaded}

\begin{boxleft}\bla{Signalflussdarstellung}
\end{boxleft}\begin{boxrightshaded}
\begin{align*}
G(p)&=\frac{\sum\limits_{i=0}^m a_i\cdot p^{-i}}{1+\sum\limits_{j=1}^n b_i\cdot p^{-j}}
\end{align*}
\end{boxrightshaded}

\section{Frequenz- und Phasengang}

\begin{boxleft}\bla{Betragsspektrum}
\end{boxleft}\begin{boxrightshaded}
\begin{align*}
\left|G(p)\right|&=\sqrt{\text{Re}^2+\text{Im}^2}
\end{align*}
\end{boxrightshaded}

\begin{boxleft}\bla{Phasengang}
\end{boxleft}\begin{boxrightshaded}
\begin{align*}
\varphi_{\underline{G}}\left(f\right)&=\arctan{\frac{\text{Im}}{\text{Re}}}
\end{align*}
\end{boxrightshaded}

\section{Transformationen}

\begin{boxleft}\bla{Laplace- und Fouriertransformation}
\end{boxleft}\begin{boxrightshaded}
\begin{align*}
\underline{X}\left(f\right)&=\int_{-\infty}^{\infty}x\left(t\right)\cdot e^{-j\omega t} \diff t=\int_{-\infty}^{\infty}x\left(t\right)\cdot \left(\cos{\omega t}-j\cdot\sin{\omega t}\right) \diff t\\
\underline{X}\left(f\right)&=\int_{-\infty}^{\infty}x\left(t\right)\cdot e^{-p t} \diff t=\int_{-\infty}^{\infty}x\left(t\right)\cdot e^{-\sigma t}\cdot \left(\cos{\omega t}-j\cdot\sin{\omega t}\right) \diff t
\end{align*}
\end{boxrightshaded}

\begin{boxleft}\bla{Z-Transformationen}
\end{boxleft}\begin{boxrightshaded}
\begin{align*}
x\left(k\right)&=\sum\limits_{k=0}^\infty x\left(m\right)\cdot\delta\left(k-m\right)\quad\laplace&& X\left(z\right)=\sum\limits_{m=0}^\infty x\left(m\right)\cdot z^{-m}\\
X\left(z\right)&=\sum\limits_{m=0}^\infty x\left(m\right)\cdot z^{-m} \quad\Leftrightarrow&& X\left(p\right)=\sum\limits_{m=0}^\infty x\left(m\right)\cdot e^{-pmt_a}\\
\varphi_z&=\omega\cdot t_a=\frac{\omega}{f_p}=2\pi\frac{f}{f_p}\\
\left|z\right|&=e^\frac{\sigma}{f_p}\\
\sigma&=\frac{\omega_p}{2\pi}\cdot \ln{\left|z\right|}\\
\omega&=\omega_p\frac{\varphi_z}{2\pi}
\end{align*}
\end{boxrightshaded}

\section{FIR-Filter}

\begin{boxleft}\bla{Darstellung im Zeitbereich}
\end{boxleft}\begin{boxrightshaded}
\begin{align*}
x_\delta\left(t\right)&=\sum\limits_{m=0}^\infty x\left(mt_a\right)\cdot \delta\left(t-mt_a\right)\\
x_\delta\left(k\right)&=\sum\limits_{m=0}^\infty x\left(m\right)\cdot \delta\left(k-m\right)\\
g\left(t\right)&=A\sum\limits_{m=0}^{M-1}a_m\delta \left(t-mt_a\right)
\end{align*}
\end{boxrightshaded}

\begin{boxleft}\bla{Darstellung im Z-Bereich}
\end{boxleft}\begin{boxrightshaded}
\begin{align*}
G\left(z\right)&=A\sum\limits_{m=0}^{M-1}a_mz^{-m}\\
&=Aa_0\frac{\sum\limits_{m=0}^{M-1}\frac{a_m}{a_0}z^{-m}}{z^{M-1}}
\end{align*}
\end{boxrightshaded}

\begin{boxleft}\bla{Frequenzgang}
\end{boxleft}\begin{boxrightshaded}
\begin{align*}
G\left(f\right)&=A\sum\limits_{m=0}^{M-1}a_me^{-j2\pi fmt_a}
\end{align*}
\end{boxrightshaded}


\section{IIR-Filter}

\begin{boxleft}\bla{Darstellung im Z-Bereich}
\end{boxleft}\begin{boxrightshaded}
\begin{align*}
G\left(z\right)&=A\cdot\frac{\sum\limits_{m=0}^{M-1}a_mz^{-m}}{1+\sum\limits_{n=1}^{N-1}b_nz^{-n}}
\end{align*}
\end{boxrightshaded}

\section{Fensterfunktion}

\begin{boxleft}\bla{Zeit- und Frequenzbereich}
\end{boxleft}\begin{boxrightshaded}
\begin{align*}
x_w\left(t\right)&=x\left(t\right)\cdot w\left(t\right)\\
X_w\left(f\right)&=X\left(f\right)\ast W\left(f\right)
\end{align*}
\end{boxrightshaded}

\begin{boxleft}\bla{Rechtecksfenster}
\end{boxleft}\begin{boxrightshaded}
\begin{align*}
w\left(t\right)&=\operatorname{rect}_T\left(\frac{t-\frac{T}{2}}{T}\right)
\end{align*}
\end{boxrightshaded}

\begin{boxleft}\bla{Dreiecksfenster}
\end{boxleft}\begin{boxrightshaded}
\begin{align*}
w\left(t\right)&=\operatorname{\Lambda}_T\left(\frac{t-\frac{T}{2}}{\frac{T}{2}}\right)
\end{align*}
\end{boxrightshaded}

\begin{boxleft}\bla{Hanningfenster}
\end{boxleft}\begin{boxrightshaded}
\begin{align*}
w\left(t\right)&=
\begin{dcases*}
  \cos^2\left(\frac{\pi\left(t-\frac{T}{2}\right)}{T}\right)\quad 0\leq t\leq T\\
  0\quad\text{sonst} 
\end{dcases*}
\end{align*}
\end{boxrightshaded}

\begin{boxleft}\bla{Hammingfenster}
\end{boxleft}\begin{boxrightshaded}
\begin{align*}
w\left(t\right)&=
\begin{dcases*}
  0,54-0,46\cos\left(\frac{2\pi t}{T}\right)\quad 0\leq t\leq T\\
  0\quad\text{sonst} 
\end{dcases*}
\end{align*}
\end{boxrightshaded}

\begin{boxleft}\bla{Blackmanfenster}
\end{boxleft}\begin{boxrightshaded}
\begin{align*}
w\left(t\right)&=
\begin{dcases*}
  0,42-0,5\cos\left(\frac{2\pi t}{T}\right)+0,08\cos\left(\frac{4\pi t}{T}\right)\quad 0\leq t\leq T\\
  0\quad\text{sonst} 
\end{dcases*}
\end{align*}
\end{boxrightshaded}

\section{Gruppenlaufzeit}

\begin{boxleft}\bla{Gruppenlaufzeit}
\end{boxleft}\begin{boxrightshaded}
\begin{align*}
\tau_{Gr}\left(\omega\right)&=-\frac{\diff\varphi_G\left(\omega\right)}{\diff \omega}
\end{align*}
\end{boxrightshaded}

\begin{boxleft}\bla{Verschiebung}
\end{boxleft}\begin{boxrightshaded}
\begin{align*}
\varphi_{\text{neu}}&=\varphi_{\text{alt}}\left(f\right)-2\pi t_0 f\\
\tau_{Gr}\left(\omega\right)&=-\frac{\diff\varphi_{\text{alt}}\left(\omega\right)}{\diff \omega}+t_0
\end{align*}
\end{boxrightshaded}

\section{Korrespodenz}

\begin{boxleft}\bla{Korrespodenz}
\end{boxleft}\begin{boxrightshaded}
\begin{align*}
x\left(t\right)&=\hat{X}\mathrm{rect}_{T}\left(t\right)\quad&\laplace\quad X\left(f\right)&=\hat{X}T\cdot\mathrm{si}\left(\pi\cdot f\cdot T\right)\\
x\left(t\right)&=\hat{X}\Lambda_{T}\left(t\right)\quad&\laplace\quad X\left(f\right)&=\hat{X}T\cdot\mathrm{si}^2\left(\pi\cdot f\cdot T\right)\\
x\left(t\right)&=\hat{X}\sin\left(2\pi f_0 t\right)\quad&\laplace\quad X\left(f\right)&=\frac{j\hat{X}}{2}\left(\delta\left(f+f_0\right)-\delta\left(f-f_0\right)\right)\\
x\left(t\right)&=\hat{X}\cos\left(2\pi f_0 t\right)\quad&\laplace\quad X\left(f\right)&=\frac{\hat{X}}{2}\left(\delta\left(f+f_0\right)+\delta\left(f-f_0\right)\right)
\end{align*}
\end{boxrightshaded}

\chapter{Zuverlässigkeit} 


\begin{boxleft}\bla{Mittlere Lebensdauer }
\des[\per\hour]{\lambda}{Ausfallrate}
\des[\hour]{\tau}{Mittlere Lebensdauer}
\end{boxleft}\begin{boxrightshaded}
\begin{align*}
    \tau&= \int_0^\infty t\cdot p\left(t\right)\diff t\\
    \tau= \int_0^\infty R\left(t\right)\diff t\\
    \tau&= \left.\frac{1}{\lambda}\right|_{\lambda~=~\text{konst} } 
\end{align*}
\end{boxrightshaded}

\begin{boxleft}\bla{Abschätzung Lebensdauer}
\end{boxleft}\begin{boxrightshaded}
\begin{align*}
    \lambda&=\frac{\Delta n}{\Delta t\cdot n\left(t\right)}\\
    \Delta n&=\left.\lambda \cdot \SI{8766}{\hour}\cdot 100\right|_{n\left(t\right)=100, \Delta t = \SI{1}{\year}}
\end{align*}
\end{boxrightshaded}

\begin{boxleft}\bla{Einsatzfaktor}
\des[\hour]{t_b}{Betriebszeit}\\
\des[\hour]{t_a}{Gesamtzeit}\\
\des{d}{Einsatzfaktor}
\end{boxleft}\begin{boxrightshaded}
\begin{align*}
    d&= \lim\limits_{t_a\to\infty}\frac{t_b}{t_a}\\
    \tau_d&=\frac{\tau}{d}
\end{align*}
\end{boxrightshaded}


\begin{boxleft}\bla{Bedingte Zuverlässigkeit}
\end{boxleft}\begin{boxrightshaded}
\begin{align*}
    \tilde{R}\left(t_1,\Delta t\right) &= \frac{n\left(t_1+\Delta t\right)}{n\left(t_1\right)}\\
    \tilde{R}\left(t_1,\Delta t\right) &= \frac{R\left(t_1+\Delta t\right)}{R\left(t_1\right)}\\
    \tilde{R}\left(t_1,\Delta t\right) &= e^{-\int_{t1}^{t1+\Delta t} \lambda\left(t\right)\diff t}
\end{align*}
\end{boxrightshaded}

\begin{boxleft}\bla{Reihenschaltung}
\end{boxleft}\begin{boxrightshaded}
\begin{align*}
    R_S\left(t\right)&=\prod_{i=1}^m R_i\left(t\right)\\
    \lambda_S\left(t\right)&=\sum_{i=1}^m \lambda_i\left(t\right)\\
    \tau_S &= \int_0^\infty R_S\left(t\right)
\end{align*}
\end{boxrightshaded}


\begin{boxleft}\bla{Parallelschaltung}
\end{boxleft}\begin{boxrightshaded}
\begin{align*}
    P_S\left(t\right)&=\prod_{i=1}^m P_i\left(t\right)\\
    \lambda_S\left(t\right)&= \left.2\lambda \frac{1-e^{-\lambda\cdot t}}{2-e^{-\lambda\cdot t}}\right|_{\lambda~=~\text{konst}, 2~\text{Parallel}}\\
    \lambda_S\left(t\right)&= \left.3\lambda \frac{1-e^{-\lambda\cdot t}+e^{-2\lambda\cdot t}}{3-3e^{-\lambda\cdot t}+e^{-2\lambda\cdot t}}\right|_{\lambda~=~\text{konst}, 3~\text{Parallel}}
\end{align*}
\end{boxrightshaded}


\begin{table}[htbp]
    \begin{tabularx}{\linewidth}{|>{$$}p{2cm}<{$$}|>{$$}X<{$$}|>{$$}X<{$$}|>{$$}X<{$$}|>{$$}X<{$$}|>{$$}X<{$$}|}\hline
        \rowcolor{ lgray} - & R\left(t\right) & P\left(t\right) & p\left(t\right)& \lambda\left(t\right)\\\hline
        R\left(t\right) & - & 1-P\left(t\right) & \int_t^\infty p\left(t\right)\diff t & e^{\int_0^t \lambda\left(t\right)\diff t}\\
        P\left(t\right) & 1-R\left(t\right) & - & \int_0^t p\left(t\right)\diff t & 1-e^{\int_0^t \lambda\left(t\right)\diff t}\\
        p\left(t\right) & -\frac{\diff R\left(t\right)}{\diff t} & \frac{\diff P\left(t\right)}{\diff t} & - & \lambda\left(t\right)\cdot e^{\int_0^t \lambda\left(t\right)\diff t} \\
        \lambda\left(t\right) & -\frac{1}{R\left(t\right)}\frac{\diff R\left(t\right)}{\diff t} & \frac{1}{1-P\left(t\right)}\frac{\diff P\left(t\right)}{\diff t}& \frac{p\left(t\right)}{\int_t^\infty p\left(t\right)\diff t} & - \\\hline
    \end{tabularx}
    \caption{Umrechnungstabelle}
\end{table}
\end{document}
