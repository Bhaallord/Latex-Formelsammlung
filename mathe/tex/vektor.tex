\section{Vektorrechnung}
\subsection{Grundlagen}

\begin{boxleft}\bla{Darstellung}
\end{boxleft}\begin{boxrightshaded}
\begin{align} 
\vv{a}	&=\vv{a}_x+\vv{a}_y+\vv{a}_z\\
	&=a_x\vv{e}_x+a_y\vv{e}_y+a_y\vv{e}_y\\
	&=\begin{pmatrix} a_x\\ a_y\\a_z\end{pmatrix}
\end{align}\end{boxrightshaded}

\begin{boxleft}\bla{2 Punkt Vektor}
\end{boxleft}\begin{boxrightshaded}
\begin{align} 
\vv{P_1P_2} &=\begin{pmatrix} x_2-x_1\\ y_2-y_1\\z_2-z_1\end{pmatrix}
\end{align}\end{boxrightshaded}

\begin{boxleft}\bla{Betrag}
\end{boxleft}\begin{boxrightshaded}
\begin{align} 
|\vv{a}|	&=a\\
		&=\sqrt{a_x^2+a_y^2+a_z^2}\\
		&=\sqrt{\vv{a}\circ\vv{a}}
\end{align}\end{boxrightshaded}

\begin{boxleft}\bla{Richtungswinkel}
\end{boxleft}\begin{boxrightshaded}
\begin{align} 
\cos \alpha &= \frac{a_x}{|\vv{a}|}\\
\cos \beta &= \frac{a_y}{|\vv{a}|}\\
\cos \gamma &= \frac{a_z}{|\vv{a}|}\\
1&=\cos^2\alpha+\cos^2\beta+\cos^2\gamma
\end{align}\end{boxrightshaded}

\subsection{Vektoroperationen}

\begin{boxleft}\bla{Addition und Subtraktion}
\end{boxleft}\begin{boxrightshaded}
\begin{align} 
\vv{a}\pm\vv{b}&= \begin{pmatrix}a_x\pm b_x\\a_y\pm b_y\\a_z\pm b_z\end{pmatrix}
\end{align}\end{boxrightshaded}

\begin{boxleft}\bla{Multiplikation mit einem Skalar}
\end{boxleft}\begin{boxrightshaded}
\begin{align} 
a\cdot\vv{b}&= \begin{pmatrix}ab_x\\ ab_y\\ab_z\end{pmatrix}
\end{align}\end{boxrightshaded}

\begin{boxleft}\bla{Einheitsvektor}
\end{boxleft}\begin{boxrightshaded}
\begin{align} 
\vv{e}_a&=\frac{\vv{a}}{|\vv{a}|}= \begin{pmatrix}a_x/|\vv{a}|\\ a_y/|\vv{a}|\\a_z/|\vv{a}|\end{pmatrix}
\end{align}\end{boxrightshaded}

\begin{boxleft}\bla{Skalarprodukt}
\end{boxleft}\begin{boxrightshaded}
\begin{align} 
\vv{a}\circ\vv{b} &= \begin{pmatrix}a_x\\ a_y\\a_z\end{pmatrix}\circ\begin{pmatrix}b_x\\ b_y\\b_z\end{pmatrix}=a_xb_x+a_yb_y+a_zb_z\\
		  &=|\vv{a}|\cdot|\vv{b}|\cdot \cos\angle(\vv{a},\vv{b})
\end{align}\end{boxrightshaded}

\begin{boxleft}\bla{Kreuzprodukt}
\destext{$|\vv{a}\times\vv{b}|$ Fläche des Parallelograms $\vv{a},\vv{b}$}\\
\destext{$\vv{a}\times\vv{b} \perp \vv{a} \land \vv{a}\times\vv{b} \perp \vv{b}$}
\end{boxleft}\begin{boxrightshaded}
\begin{align} 
\vv{a}\times\vv{b} &= \begin{pmatrix}a_x\\ a_y\\a_z\end{pmatrix}\times\begin{pmatrix}b_x\\ b_y\\b_z\end{pmatrix}=\begin{pmatrix}a_yb_z-a_zb_y\\ a_zb_x-a_xb_z\\a_xb_y-a_yb_x\end{pmatrix}\\
		  &=\begin{vmatrix}\vv{e}_x&\vv{e}_y&\vv{e}_z\\a_x&a_y&a_z\\b_x&b_y&b_z\end{vmatrix}
\end{align}\end{boxrightshaded}

\begin{boxleft}\bla{Spatprodukt}
\destext{$\vv{a}\circ(\vv{b}\times\vv{c})$ Volumen des Parallelpiped $\vv{a},\vv{b},\vv{c}$}
\end{boxleft}\begin{boxrightshaded}
\begin{align} 
[\vv{a}\vv{b}\vv{c}]  &=\vv{a}\circ(\vv{b}\times\vv{c})\\
		      &=a_x(b_yc_z-b_zc_y)+a_y(b_zc_x-b_xc_z)+a_z(b_xc_y-b_yc_x)\\
		      &=\begin{vmatrix}a_x&a_y&a_z\\b_x&b_y&b_z\\c_x&c_y&c_z\end{vmatrix}
\end{align}\end{boxrightshaded}

\begin{boxleft}\bla{Schnittwinkel}
\end{boxleft}\begin{boxrightshaded}
\begin{align} 
\cos\angle(\vv{a},\vv{b})&=\frac{\vv{a}\circ\vv{b}}{|\vv{a}|\cdot|\vv{b}|}
\end{align}\end{boxrightshaded}

\begin{boxleft}\bla{Projektion}
\end{boxleft}\begin{boxrightshaded}
\begin{align} 
\vv{a}_b&=\left(\frac{\vv{a}\circ\vv{b}}{|\vv{a}|^2}\right)\vv{a}=(\vv{b}\circ\vv{e}_a)\vv{e}_a
\end{align}\end{boxrightshaded}

\subsection{Geraden}


\begin{boxleft}\bla{Geradegleichung}
\des[]{\vv{r}_1}{Ortsvektor (Verschiebung von Ursprung)}\\
\des[]{\vv{a}}{Richtungsvektor}
\end{boxleft}\begin{boxrightshaded}
\begin{align} 
\vv{r}(t) &=\vv{r}_1+t\vv{a}\\
	  &=\vv{r}_1+t(\vv{r}_2-\vv{r}_1)
\end{align}\end{boxrightshaded}

\begin{boxleft}\bla{Abstand eines Punktes von einer Geraden}
\des[]{\vv{r}_1}{Ortsvektor (Verschiebung von Ursprung)}\\
\des[]{\vv{a}}{Richtungsvektor}\\
\des[]{\vv{OP}}{Ortsvektor des Punktes P}
\end{boxleft}\begin{boxrightshaded}
\begin{align} 
\vv{r}(t) &=\vv{r}_1+t\vv{a}\\
d&=\frac{|\vv{a}\times\left(\vv{OP}-\vv{r}_1\right)|}{\vv{a}}
\end{align}\end{boxrightshaded}

\begin{boxleft}\bla{Abstand zweier paralleler Geraden}
\des[]{\vv{r}_1}{Ortsvektor der ersten Gerade}\\
\des[]{\vv{r}_2}{Ortsvektor der zweiten Gerade}\\
\des[]{\vv{a}_1}{Richtungsvektor der Geraden}
\end{boxleft}\begin{boxrightshaded}
\begin{align} 
\vv{r}(t) &=\vv{r}_1+t\vv{a}_1\\
\vv{g}(t) &=\vv{r}_2+t\vv{a}_1\\
d&=\frac{|\vv{a}_1\times\left(\vv{r}_2-\vv{r}_1\right)|}{\vv{a}_1}
\end{align}\end{boxrightshaded}


\begin{boxleft}\bla{Abstand zweier windschiefen Geraden}
\des[]{\vv{r}_1}{Ortsvektor der ersten Gerade}\\
\des[]{\vv{r}_2}{Ortsvektor der zweiten Gerade}\\
\des[]{\vv{a}_1}{Richtungsvektor der ersten Geraden}\\
\des[]{\vv{a}_2}{Richtungsvektor der zweiten Geraden}
\end{boxleft}\begin{boxrightshaded}
\begin{align} 
\vv{r}(t) &=\vv{r}_1+t\vv{a}_1\\
\vv{g}(t) &=\vv{r}_2+t\vv{a}_2\\
d&=\frac{|\vv{a}_1\circ\left(\vv{a}_2\times\left(\vv{r}_2-\vv{r}_1\right)\right)|}{\vv{a}_1\times\vv{a}_2}
\end{align}\end{boxrightshaded}

\subsection{Ebenen}

\begin{boxleft}\bla{Ebenengleichung}
\des[]{\vv{r}_1}{Ortsvektor der Ebenen}\\
\des[]{\vv{a}_1}{Erster Richtungsvektor}\\
\des[]{\vv{a}_2}{Zweiter Richtungsvektor}
\end{boxleft}\begin{boxrightshaded}
\begin{align} 
\vv{r}(t,s) &=\vv{r}_1+t\vv{a}_1+s\vv{a}_2\\
	    &=\vv{r}_1+t(\vv{r}_2-\vv{r}_1)+s(\vv{r}_3-\vv{r}_1)
\end{align}\end{boxrightshaded}

\begin{boxleft}\bla{Normalenvektor}
\des[]{\vv{n}}{Normalenvektor}\\
\des[]{\vv{r}_1}{Ortsvektor der Normalen}\\
\des[]{\vv{r}}{$(x,y,z)^T$}
\end{boxleft}\begin{boxrightshaded}
\begin{align} 
\vv{n}&=\vv{a}_1\times\vv{a}_2

\end{align}\end{boxrightshaded}


\begin{boxleft}\bla{Parameterfreie Darstellung}
\des[]{\vv{n}}{Normalenvektor}
\end{boxleft}\begin{boxrightshaded}
\begin{align} 

\vv{r}(t,s) &=\vv{r}_1+t\vv{a}_1+s\vv{a}_2\\
\vv{r}\circ(\vv{a}_1\times\vv{a}_2)&=\vv{r}_1\circ(\vv{a}_1\times\vv{a}_2)+t\vv{a}_1\circ(\vv{a}_1\times\vv{a}_2)\\
&\qquad +s\vv{a}_2\circ(\vv{a}_1\times\vv{a}_2)\\
\vv{r}\circ\vv{n}&=\vv{r}_1\circ\vv{n}+0+0\\
\vv{n}\circ\left(\vv{r}-\vv{r}_1\right)=0
\end{align}\end{boxrightshaded}

\begin{boxleft}\bla{Normierter Normalenvektor}
\end{boxleft}\begin{boxrightshaded}
\begin{align} 
\vv{e}_n&=\frac{\vv{a}_1\times\vv{a}_2}{|\vv{a}_1\times\vv{a}_2|}
\end{align}\end{boxrightshaded}

\begin{boxleft}\bla{Hesseschen Normalform}
\end{boxleft}\begin{boxrightshaded}
\begin{align} 
0&=\frac{Ax+By+Cz+D}{\sqrt{A^2+B^2+C^2}}
\end{align}\end{boxrightshaded}


\begin{boxleft}\bla{Abstand eines Punktes von einer Ebene}
\des[]{\vv{n}}{Normalenvektor}\\
\des[]{\vv{r}_1}{Ortsvektor der Normalen}\\
\des[]{\vv{OP}}{Ortsvektor des Punktes P}\\
\des[]{p_i}{Koordinaten des Punktes P}
\end{boxleft}\begin{boxrightshaded}
\begin{align} 
d&=\frac{|\vv{n}\times\left(\vv{OP}-\vv{r}_1\right)|}{\vv{n}}\\
d&=\frac{Ap_1+Bp_2+Cp_3+D}{\sqrt{A^2+B^2+C^2}}
\end{align}\end{boxrightshaded}


\begin{boxleft}\bla{Abstand eines Geraden von einer Ebene}
\des[]{\vv{n}}{Normalenvektor}\\
\des[]{\vv{r}_1}{Ortsvektor der Normalen}\\
\des[]{\vv{r}_G}{Ortsvektor der Geraden}\\
\des[]{r_{Gi}}{Koordinaten eines Geraden Punktes}
\end{boxleft}\begin{boxrightshaded}
\begin{align} 
\vv{r}(t) &=\vv{r}_G+t\vv{a}_1\\
d&=\frac{|\vv{n}\times\left(\vv{r}_G-\vv{r}_1\right)|}{\vv{n}}\\
d&=\frac{Ar_{G1}+Br_{G2}+Cr_{G3}+D}{\sqrt{A^2+B^2+C^2}}
\end{align}\end{boxrightshaded}


\begin{boxleft}\bla{Abstand zweier paralleler Ebenen}
\des[]{\vv{n}}{Normalenvektor}
\end{boxleft}\begin{boxrightshaded}
\begin{align} 
\vv{r}(t,s) &=\vv{r}_1+t\vv{a}_1+s\vv{a}_2\\
\vv{g}(t,s) &=\vv{r}_2+t\vv{a}_3+s\vv{a}_4\\
d&=\frac{|\vv{n}\times\left(\vv{r}_1-\vv{r}_2\right)|}{\vv{n}}
\end{align}\end{boxrightshaded}


\begin{boxleft}\bla{Schnittwinkel zweier Ebenen}
\destext{$\angle$ Ebenen = $\angle(\vv{n}_1,\vv{n}_2)$}
\end{boxleft}\begin{boxrightshaded}
\begin{align} 
\cos\angle(\vv{n}_1,\vv{n}_2)&=\frac{\vv{n}_1\circ\vv{n}_2}{|\vv{n}_1|\cdot|\vv{n}_2|}
\end{align}\end{boxrightshaded}

\begin{boxleft}\bla{Durchstoßpunkt}
\des[]{\vv{n}}{Normalenvektor}\\
\des[]{\vv{r}_1}{Ortsvektor der Normalen}\\
\des[]{\vv{r}_G}{Ortsvektor der Geraden}\\
\des[]{\vv{r}_s}{Ortsvektor des Schnittpunktes}
\end{boxleft}\begin{boxrightshaded}
\begin{align} 
\vv{r}(t) &=\vv{r}_G+t\vv{a}\\
\vv{r}_s&=\vv{r}_G+\frac{\vv{n}\circ\left(\vv{r}_1-\vv{r}_G\right)}{\vv{n}\circ\vv{a}}\vv{a}\\
\varphi&=\arcsin\left(\frac{|\vv{n}\circ\vv{a}|}{|\vv{n}|\cdot|\vv{a}|}\right)
\end{align}\end{boxrightshaded}
