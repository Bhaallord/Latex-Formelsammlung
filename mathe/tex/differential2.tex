 
\section{Differential- und Integralrechnung mit mehreren Variablen}
\subsection{Differentialrechnung}

\begin{boxleft}\bla{Aleitung}
\end{boxleft}\begin{boxrightshaded}
\begin{align*}
y&=f(x_1,x_2,\dots,x_3)\\
\frac{\partial y}{\partial x_1}&=y_{x_1}&&\text{Alles bis auf $x_1$ ist konstant beim ableiten}\\
\frac{\partial y}{\partial x_n}&=y_{x_n}&&\text{Alles bis auf $x_n$ ist konstant beim ableiten}\\
\frac{\partial^2 y}{\partial x_1^2}&=y_{x_1x_1}&&\text{Alles bis auf $x_1$ ist konstant beim ableiten}\\
y_{x_1x_2}&=y_{x_2x_1}
\end{align*}
\end{boxrightshaded}

\begin{boxleft}\bla{Tangentialebene}
\des{x_0}{Entwicklungspunkt der Ebene}\\
\des{y_0}{Entwicklungspunkt der Ebene}
\end{boxleft}\begin{boxrightshaded}
\begin{align*}
z-z_0&=f_x\left(x_0;y_0\right)\cdot\left(x-x_0\right)+f_y\left(x_0;y_0\right)\cdot\left(y-y_0\right)
\end{align*}
\end{boxrightshaded}

\begin{boxleft}\bla{Totales Differential}
\end{boxleft}\begin{boxrightshaded}
\begin{align*}
\diff z&=f_x\cdot\diff x+f_y\cdot\diff y
\end{align*}
\end{boxrightshaded}

\begin{boxleft}\bla{Extrema}
\end{boxleft}\begin{boxrightshaded}
\begin{align*}
&f_x(x_0,y_0)=0&&f_y(x_0,y_0)=0\\
&f_{xx}(x_0;y_0)<0&&\text{Maximum}\\
&f_{xx}(x_0;y_0)>0&&\text{Minimum}\\
&\begin{vmatrix}f_{xx}(x_0;y_0)&f_{xy}(x_0;y_0)\\f_{xy}(x_0;y_0)&f_{yy}(x_0;y_0)\end{vmatrix}>0
\end{align*}
\end{boxrightshaded}

\begin{boxleft}\bla{Sattelpunkt}
\end{boxleft}\begin{boxrightshaded}
\begin{align*}
&f_x(x_0,y_0)=0&&f_y(x_0,y_0)=0\\
&\begin{vmatrix}f_{xx}(x_0;y_0)&f_{xy}(x_0;y_0)\\f_{xy}(x_0;y_0)&f_{yy}(x_0;y_0)\end{vmatrix}<0
\end{align*}
\end{boxrightshaded}


\begin{boxleft}\bla{Richtungsableitung}
\end{boxleft}\begin{boxrightshaded}
\begin{align*}
\frac{\partial z}{\partial \vv{a}}&=\frac{1}{\sqrt{a_x^2+a_y^2}}\cdot\left(a_xz_x+a_yz_y\right)\\
\frac{\partial z}{\partial \alpha}&=z_x\cos\alpha+z_y\sin\alpha\\
\frac{\partial z}{\partial \alpha}&=\vv{e_a}\cdot \grad\left(z\right)
\end{align*}
\end{boxrightshaded}

\subsection{Mehrfachintegral}

\begin{boxleft}\bla{Polarkordinaten}
\end{boxleft}\begin{boxrightshaded}
\begin{align*}
x&=x_0+r\cos\varphi&y&=y_0+r\sin\varphi
\end{align*}
\end{boxrightshaded}

\begin{boxleft}\bla{Volumen}
\end{boxleft}\begin{boxrightshaded}
\begin{align*}
\iiint_{V}\diff V&=\int_x\int_y\int_z \diff z\diff y\diff x\\
\iiint_{V}\diff V&=\int_r\int_\varphi\int_z r\diff z\diff r \diff \varphi\\
\end{align*}
\end{boxrightshaded}

\begin{boxleft}\bla{Fläche}
\end{boxleft}\begin{boxrightshaded}
\begin{align*}
A&=\iint_{(A)}\diff A
\end{align*}
\end{boxrightshaded}

\begin{boxleft}\bla{Masse}
\end{boxleft}\begin{boxrightshaded}
\begin{align*}
m&=\iint_{(A)}\rho(x,y)\diff x\diff y\\
m&=\iint_{(A)}\rho(r,\varphi)r\diff r\diff \varphi\\
m&=\iiint_{(V)}\rho(x,y)\diff z\diff x\diff y\\
m&=\iiint_{(V)}\rho(r,\varphi)r\diff z\diff r\diff \varphi
\end{align*}
\end{boxrightshaded}

\begin{boxleft}\bla{Statische Moment}
\des{M_x}{Moment bezüglich x-Achse}\\
\des{M_y}{Moment bezüglich y-Achse}\\
\end{boxleft}\begin{boxrightshaded}
\begin{align*}
M_x&=\iint_{(A)}y\rho(x,y)\diff x\diff y\\
M_x&=\iint_{(A)}y_0+r\sin\varphi\rho(r,\varphi)r\diff r\diff \varphi\\
M_y&=\iint_{(A)}x\rho(x,y)\diff x\diff y\\
M_y&=\iint_{(A)}x_0+r\cos\varphi\rho(r,\varphi)r\diff r\diff \varphi
\end{align*}
\end{boxrightshaded}

\begin{boxleft}\bla{Schwerpunkt}
\end{boxleft}\begin{boxrightshaded}
\begin{align*}
x_s&=\frac{M_y}{m}\\
y_s&=\frac{M_x}{m}\\
\end{align*}
\end{boxrightshaded}

\begin{boxleft}\bla{Trägheitsmoment}
\end{boxleft}\begin{boxrightshaded}
\begin{align*}
I_x&=\iint_{(A)}y^2\rho(x,y)\diff x\diff y\\
I_x&=\iint_{(A)}\left(y_0+r\sin\varphi\right)^2\rho(r,\varphi)r\diff r\diff \varphi\\
I_y&=\iint_{(A)}x^2\rho(x,y)\diff x\diff y\\
I_y&=\iint_{(A)}\left(x_0+r\cos\varphi\right)^2\rho(r,\varphi)r\diff r\diff \varphi
\end{align*}
\end{boxrightshaded}

\begin{boxleft}\bla{Polares Trägheitsmoment}
\end{boxleft}\begin{boxrightshaded}
\begin{align*}
I_x&=\iint_{(A)}\left(y^2+x^2\right)\rho(x,y)\diff x\diff y\\
I_x&=\iint_{(A)}\left(\left(y_0+r\sin\varphi\right)^2+\left(x_0+r\cos\varphi\right)^2\right)\rho(r,\varphi)r\diff r\diff \varphi
\end{align*}
\end{boxrightshaded}

\begin{boxleft}\bla{Kugelkoordinaten}
\end{boxleft}\begin{boxrightshaded}
\begin{align*}
V&=\int_r\int_\vartheta\int_\varphi r^2\sin\vartheta\diff \varphi \diff \vartheta \diff r
\end{align*}
\end{boxrightshaded}

