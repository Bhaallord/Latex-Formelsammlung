\section{Differntialrechnung}

\subsection{Erste Ableitung der elementaren Funktionen}
\begin{boxleft}
  \bla{Potenzfunktion}
\end{boxleft}\begin{boxrightshaded}
 \begin{align}
  &x^n& 	&n\cdot x^{n-1}
 \end{align}
\end{boxrightshaded}

\begin{boxleft}
  \bla{Exponentialfunktionen}
\end{boxleft}\begin{boxrightshaded} \begin{align}
  &e^x& 	&e^x\\
  &a^x& 	&\ln a\cdot a^x
 \end{align}
\end{boxrightshaded}

\begin{boxleft}
  \bla{Logarithmusfunktionen}
\end{boxleft}\begin{boxrightshaded}
 \begin{align} 
  &\ln x& &\frac{1}{x}\\
  &\log_a x&	&\frac{1}{(\ln a)\cdot x}
 \end{align}
\end{boxrightshaded}

\begin{boxleft}
  \bla{Trigonometrische Funktionen}
\end{boxleft}\begin{boxrightshaded}
 \begin{align} 
  &\sin x& 	&\cos x \\
  &\cos x& 	&-\sin x\\
  &\tan x&	&\frac{1}{\cos^2 x}\\
  &\tan x&	&1+\tan^2 x
\end{align}\end{boxrightshaded}
 
\begin{boxleft}
  \bla{Arcusfunktionen}
\end{boxleft}\begin{boxrightshaded}
 \begin{align} 
  &\arcsin x& &\frac{1}{\sqrt{1-x^2}}\\
  &\arccos x& &\frac{-1}{\sqrt{1-x^2}}\\
  &\arctan x& &\frac{1}{1-x^2}
\end{align}\end{boxrightshaded}

\begin{boxleft}
  \bla{Hyperbelfunktionen}
\end{boxleft}\begin{boxrightshaded}
 \begin{align} 
  &\sinh x& &\cosh x\\
  &\cosh x& &\sinh x\\
  &\tanh x& &\frac{1}{\cosh^2 x}\\
  &\tanh x& &1+\tanh^2 x
\end{align}\end{boxrightshaded}
          
\subsection{Rechenregeln}

\begin{boxleft}
  \bla{Faktorregel}
\end{boxleft}\begin{boxrightshaded}
 \begin{align} 
\frac{\diff }{\diff x}\left(C\cdot f(x)\right)=C\cdot f'(x)
 \end{align}\end{boxrightshaded}

\begin{boxleft}
  \bla{Summenregel}
\end{boxleft}\begin{boxrightshaded}
 \begin{align} 
\frac{\diff }{\diff x}\left(g(x) + f(x)\right)=g'(x)+f'(x)
 \end{align}\end{boxrightshaded}

\begin{boxleft}
  \bla{Produktregel}
\end{boxleft}\begin{boxrightshaded}
 \begin{align} 
\frac{\diff }{\diff x}&\left(g(x)\cdot f(x)\right)=g'(x)\cdot f(x)+g(x)\cdot f'(x)\\
\frac{\diff }{\diff x}&\left(h(x)\cdot g(x)\cdot f(x)\right)=h'\cdot g\cdot f+h\cdot g'\cdot f+h\cdot g\cdot f'
 \end{align}\end{boxrightshaded}
            
\begin{boxleft}
  \bla{Quotientenregel}
\end{boxleft}\begin{boxrightshaded}
 \begin{align} 
\frac{\diff }{\diff x}\left(\frac{g(x)}{f(x)}\right)&=\frac{g'(x)\cdot f(x)-g(x)\cdot f'(x)}{f(x)^2}
 \end{align}\end{boxrightshaded}
            
\begin{boxleft}
  \bla{Kettenregel}
\end{boxleft}\begin{boxrightshaded}
 \begin{align} 
\frac{\diff }{\diff x}\left(g\left( f(x\right)\right)&=g'(f)\cdot f'(x)
 \end{align}\end{boxrightshaded}
                 
\begin{boxleft}
  \bla{Logarithmische Ableitungen}
\end{boxleft}\begin{boxrightshaded}
 \begin{align} 
\frac{\diff }{\diff x}y&=f(x)\\
\frac{1}{y}y'&=\frac{\diff }{\diff x}\ln{f(x)}
 \end{align}\end{boxrightshaded}
                   
\subsection{Fehlerrechnung}

\begin{boxleft}\bla{Absolute Fehler}
\des[]{\Delta x}{Absoluter Fehler der Eingangsgröße}\\
\des[]{\Delta y}{Absoluter Fehler der Ausgangsgröße}
\end{boxleft}\begin{boxrightshaded}
\begin{align} 
\Delta y=f(x+\Delta x)-f(x)
\end{align}\end{boxrightshaded}
         
\begin{boxleft}\bla{Relativer Fehler}
\des[]{\delta x}{Relativer Fehler der Eingangsgröße in $\%$}\\
\des[]{\delta y}{Relativer Fehler der Ausgangsgröße in $\%$}
\end{boxleft}\begin{boxrightshaded}
\begin{align} 
\delta x&=\frac{\Delta x}{x}\\
\delta y&=\frac{\Delta y}{y}\\
\Delta y&=f'(x)\cdot \Delta x\\
\delta y&=\frac{x\cdot f'(x)}{f(x)}\delta x
\end{align}\end{boxrightshaded}
         
\subsection{Linearisierung und Taylor-Polynome}

\begin{boxleft}\bla{Tangentengleichung}
\des[]{x_0}{Punkt an denn das Polynome entwickelt wird}
\end{boxleft}\begin{boxrightshaded}
\begin{align} 
y_T(x)&=f(x_0)+f'(x_0)(x-x_0)
\end{align}\end{boxrightshaded}

\begin{boxleft}\bla{Taylor Polynome}
\des[]{x_0}{Punkt an denn das Polynome entwickelt wird}\\
\des[]{R_n}{Restglied}
\end{boxleft}\begin{boxrightshaded}
\begin{align} 
y(x)&=f(x_0)+f'(x_0)(x-x_0)+\frac{f''(x_0)}{2!}(x-x_0)^2+\dots+\frac{f^{(n)}(x_0)}{n!}(x-x_0)^n+R_n(x)\\
    &=\sum_{i=0}^n\frac{f^{(i)}}{i!}(x-x_0)^i+R_n(x)
\end{align}\end{boxrightshaded}


\begin{boxleft}\bla{Restglied}
\des[]{x_0}{Punkt an denn das Polynome entwickelt wird}\\
\des[]{c}{$x_0<c<x$, wenn $x_0<x$}\\
\des[]{c}{$x_0>c>x$, wenn $x_0>x$}
\end{boxleft}\begin{boxrightshaded}
\begin{align} 
R_n(x)=\frac{f^{(n+1)}(c)}{(n+1)!}(x-x_0)^{n+1}
\end{align}\end{boxrightshaded}

\subsection{Grenzwertregel von Bernoulli und de l`Hospital}
\begin{boxleft}\bla{de l`Hospital}
\destext{Gilt nur wenn $\lim_{x \to x_0} f(x)$ gleich $\frac{0}{0}$ oder $\frac{\infty}{\infty}$ ist}
\end{boxleft}\begin{boxrightshaded}
\begin{align} 
 \lim_{x \to x_0}\frac{f(x)}{g(x)}=\lim_{x \to x_0}\frac{f'(x)}{g'(x)}
\end{align}\end{boxrightshaded}

\subsection{Differentielle Kurvenuntersuchung}


\begin{boxleft}\bla{Normale der Kurve}
\end{boxleft}\begin{boxrightshaded}
\begin{align} 
y_N(x)&=f(x_0)-\frac{1}{f'(x)}\left(x-x_0\right)
\end{align}\end{boxrightshaded}

\begin{boxleft}\bla{Monotonie-Verhalten}
\end{boxleft}\begin{boxrightshaded}
\begin{align} 
f'(x)&>0&&\text{Monoton wachsend}\\
f'(x)&<0&&\text{Monoton fallend} 
\end{align}\end{boxrightshaded}

\begin{boxleft}\bla{Krümmung-Verhalten}
\end{boxleft}\begin{boxrightshaded}
\begin{align} 
f''(x)&>0&&\text{Linkskrümmung(konvex)}\\
f''(x)&<0&&\text{Rechtskrümmung(konkav)} 
\end{align}\end{boxrightshaded}

\begin{boxleft}\bla{Ableitung Polarkordinaten}
\des[]{\dot{r}}{Ableitung nach $\varphi$}\\
\des[]{\ddot{r}}{Zweite Ableitung nach $\varphi$}
\end{boxleft}\begin{boxrightshaded}
\begin{align} 
y(\varphi)&=r(\varphi)\sin\varphi\\
x(\varphi)&=r(\varphi)\cos\varphi\\
y'&=\frac{\diff y}{\diff x}=\frac{r'\sin\varphi+r\cos\varphi}{r'\cos\varphi-r\sin\varphi}\\
y''&=\frac{\diff^2 y}{\diff x^2}=\frac{2(r')^2-r\cdot r''+r^2}{\left(r'\cos\varphi-r\sin\varphi\right)^3}
\end{align}\end{boxrightshaded}

\begin{boxleft}\bla{Ableitung Parameterform}
\des[]{\dot{x}}{Ableitung nach t}\\
\des[]{\dot{y}}{Ableitung nach t}
\end{boxleft}\begin{boxrightshaded}
\begin{align} 
y&=y(t)\\
x&=x(t)\\
y'&=\frac{\diff y}{\diff x}=\frac{\dot{y}}{\dot{x}}\\
y''&=\frac{\diff^2 y}{\diff x^2}=\frac{\dot{x}\ddot{y}-\dot{y}\ddot{x}}{\dot{x}^3}
\end{align}\end{boxrightshaded}

\begin{boxleft}\bla{Ableitung Parameterform}
\des[]{\dot{x}}{Ableitung nach t}\\
\des[]{\dot{y}}{Ableitung nach t}
\end{boxleft}\begin{boxrightshaded}
\begin{align} 
y&=y(t)\\
x&=x(t)\\
y'&=\frac{\diff y}{\diff x}=\frac{\dot{y}}{\dot{x}}\\
y''&=\frac{\diff^2 y}{\diff x^2}=\frac{\dot{x}\ddot{y}-\dot{y}\ddot{x}}{\dot{x}^3}
\end{align}\end{boxrightshaded}

\begin{boxleft}\bla{Bogendifferential}
\destext{"Wegelement" einer Funktion}
\end{boxleft}\begin{boxrightshaded}
\begin{align} 
\diff s&=\sqrt{1+\left(f'(x)\right)^2}\cdot \diff x\\
\diff s&=\sqrt{(\dot{x})^2+(\dot{y})^2}\cdot \diff t\\
\diff s&=\sqrt{r^2+(r')^2}\cdot \diff \varphi
\end{align}\end{boxrightshaded}

\begin{boxleft}\bla{Winkeländerung}
\end{boxleft}\begin{boxrightshaded}
\begin{align} 
\tau&=\arctan y'\\
\diff \tau&=\frac{y''}{1+(y')^2}\cdot \diff x
\end{align}\end{boxrightshaded}

\begin{boxleft}\bla{Kurvenkrümmung}
\end{boxleft}\begin{boxrightshaded}
\begin{align} 
\kappa&=\frac{\diff \tau}{\diff s}\\
\kappa&=\frac{y''}{\sqrt{\left(1+(y')^2\right)^3}}\\
\kappa&=\frac{\dot{x}\ddot{y}-\dot{y}\ddot{x}}{\sqrt{\left(\dot{x}^2+\dot{y}^2\right)^3}}\\
\kappa&=\frac{2(r')^2-r\cdot r''+r^2}{\sqrt{\left(r^2+(r')^2\right)^3}}
\end{align}\end{boxrightshaded}

\begin{boxleft}\bla{Krümmungskreis}
\des[]{\rho}{Radius des Krümmungskreises}\\
\des[]{x_K}{x-Koordinaten des Kreismittelpunktes}\\
\des[]{y_K}{y-Koordinaten des Kreismittelpunktes}\\
\des[]{x_P}{x-Koordinaten des Kurvenpunktes}\\
\des[]{y_P}{y-Koordinaten des Kurvenpunktes}
\end{boxleft}\begin{boxrightshaded}
\begin{align} 
\rho&=\frac{1}{|\kappa|}\\
x_K&=x_P-y'\frac{1+(y')^2}{|y''|}\\
y_K&=y_P+\frac{1+(y')^2}{|y''|}
\end{align}\end{boxrightshaded}




