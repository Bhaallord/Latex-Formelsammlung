\chapter{Signal- und Systemtheorie}

\section{Grundsignale}

\subsection{Einheitsignale}
\begin{boxleft}\bla{Diracstoß}
\des[\per\second]{\delta\left(t\right)}{Diracstoß}
\end{boxleft}\begin{boxrightshaded}
\begin{align*}
\delta\left(t\right)&=
\begin{dcases*}
  0\si{\per\second}&$t<0$\\
\infty\si{\per\second}& $t=0$\\
  0\si{\per\second}&$t>0$
\end{dcases*}\\
\int_{-\infty}^\infty\delta\left(t\right)\diff t&=1\\
\delta\left(t\right)&=\frac{\diff \sigma\left(t\right)}{\diff t}=\frac{\diff^2 \alpha\left(t\right)}{\diff t^2}
\end{align*}
\end{boxrightshaded}


\begin{boxleft}\bla{Einheitssprungsfunktion}
\des{\sigma\left(t\right)}{Einheitssprungsfunktion}
\end{boxleft}\begin{boxrightshaded}
\begin{align*}
\sigma\left(t\right)&=
\begin{dcases*}
  0&$t<0$\\
0,5& $t=0$\\
  1&$t>0$
\end{dcases*}\\
\sigma\left(t\right)&=\int_{-\infty}^t \delta\left(t\right)\diff t=\frac{\diff \alpha\left(t\right)}{\diff t}
\end{align*}
\end{boxrightshaded}


\begin{boxleft}\bla{Einheitsanstiegsfunktion}
\des[\second]{\alpha\left(t\right)}{Einheitsanstiegsfunktion}
\end{boxleft}\begin{boxrightshaded}
\begin{align*}
\alpha\left(t\right)&=
\begin{dcases*}
  0\si{\second}&$t<0$\\
t& $t=0$
\end{dcases*}\\
\alpha\left(t\right)&=\iint_{-\infty}^t \delta\left(t\right)\diff t\diff t=\int_{-\infty}^t \sigma\left(t\right)\diff t
\end{align*}
\end{boxrightshaded}


\subsection{Weitere Grundsignale}
\begin{boxleft}\bla{Rechtecksimpuls}
\des{\operatorname{rect}_T\left(t\right)}{Rechtecksimpuls}
\end{boxleft}\begin{boxrightshaded}
\begin{align*}
\operatorname{rect}_T\left(t\right)&=
\begin{dcases*}
  1&$\left|t\right|<\frac{T}{2}$\\
0,5& $\left|t\right|=\frac{T}{2}$\\
0& $\left|t\right|>\frac{T}{2}$
\end{dcases*}
\end{align*}
\end{boxrightshaded}


\begin{boxleft}\bla{Dreiecksimpuls}
\des{\operatorname{\Lambda}_T\left(t\right)}{Dreiecksimpuls}
\end{boxleft}\begin{boxrightshaded}
\begin{align*}
\operatorname{\Lambda}_T\left(t\right)&=
\begin{dcases*}
  1+\frac{t}{T}&$-T<t<0$\\
  1-\frac{t}{T}&$0\leq t<T$\\
  0&$\left|t\right|>T$\\
\end{dcases*}
\end{align*}
\end{boxrightshaded}

\subsection{Signalveränderungen}
\begin{boxleft}\bla{Offset}
\des{X_{off}}{Offsetwert}
\end{boxleft}\begin{boxrightshaded}
\begin{align*}
x_2\left(t\right)&=x_1\left(t\right)+X_{off}
\end{align*}
\end{boxrightshaded}


\begin{boxleft}\bla{Skalierung}
\des{V}{Verstärkungsfaktor}
\end{boxleft}\begin{boxrightshaded}
\begin{align*}
x_2\left(t\right)&=V\cdot x_1\left(t\right)
\end{align*}
\end{boxrightshaded}


\begin{boxleft}\bla{Verschiebung}
\des{t_0}{Verschiebungskonstante}
\end{boxleft}\begin{boxrightshaded}
\begin{align*}
x_2\left(t\right)&= x_1\left(t-t_0\right)&&\text{$t_0>0$: Rechtsverschiebung}
\end{align*}
\end{boxrightshaded}

\begin{boxleft}\bla{Negation des Argumentes}
\end{boxleft}\begin{boxrightshaded}
\begin{align*}
x_2\left(t\right)&= x_1\left(-t\right)&&\text{Spiegelung an der Ordinate}
\end{align*}
\end{boxrightshaded}


\begin{boxleft}\bla{Negiertes und verschobenes Argument}
\des{t_0}{Verschiebungskonstante}
\end{boxleft}\begin{boxrightshaded}
\begin{align*}
x_2\left(t\right)&= x_1\left(-\left(t-t_0\right)\right)&&\text{Spiegelung bei $\frac{t_0}{2}$}
\end{align*}
\end{boxrightshaded}


\begin{boxleft}\bla{Argumentskalierung}
\end{boxleft}\begin{boxrightshaded}
\begin{align*}
x_2\left(t\right)&= x_1\left(a\cdot t\right)&&\text{$a<1$ Streckung der Funktion}
\end{align*}
\end{boxrightshaded}



\section{Signaleigenschaften}

\subsection{Energiesignale}

\begin{boxshaded}
\begin{align*}
&\text{$E=$endlich positiver Wert. $P=0$} 
\end{align*}
\end{boxshaded}

\begin{boxleft}\bla{Energie}
\des[\watt\second]{E_R}{Energie}\\
\des{E_X}{Normierte Signalenergie}
\end{boxleft}\begin{boxrightshaded}
\begin{align*}
E_R&=\int_{-\infty}^\infty u\left(t\right)\cdot i\left(t\right)\diff t\\
&=\int_{-\infty}^\infty \frac{u^2\left(t\right)}{R}\diff t\\
E_x&=m_{i2}=\int_{-\infty}^\infty x^2\left(t\right)\diff t&&\text{Normierung auf $R=1$}\\
E_x&=\sum_{k=-\infty}^{\infty}x^2\left(k\right)
\end{align*}
\end{boxrightshaded}

\begin{boxleft}\bla{Impulsfläsche}
\des[\watt\second]{E_R}{Energie}
\end{boxleft}\begin{boxrightshaded}
\begin{align*}
A_x&=m_{i1}=\int_{-\infty}^\infty x\left(t\right)\diff t\\
A_x&=\sum_{k=-\infty}^{\infty}x\left(k\right)
\end{align*}
\end{boxrightshaded}


\subsection{Leistungssignale}


\begin{boxshaded}
\begin{align*}
&\text{$E=\infty$. $P=$endlich positiver Wert.} 
\end{align*}
\end{boxshaded}

\begin{boxleft}\bla{Mittlere Signalleistung}
\des{P_x}{Mittlere Signalleistung}\\
\des{\bar{x^2}}{quadratischer Mittelwert}\\
\des{m_2}{gewöhnliches Moment 2. Ordnung}\\
\des{x_0^2}{Konstantes Signale}
\end{boxleft}\begin{boxrightshaded}
\begin{align*}
P_x&=\bar{x^2}=m_2=\lim_{T\to \infty}\int_{t_0}^{t_0+T} x^2\left(t\right)\diff t\\
&=\frac{1}{n\cdot T_p}\int_{t_0}^{t_0+n\cdot T_p} x^2\left(t\right)\diff t&&\text{Periodische Signale}\\
&=\lim_{N\to \infty}\frac{1}{N}\sum_{k=k_0}^{k_0+N-1}x^2\left(k\right)\\
&=\frac{1}{N_p}\sum_{k=k_0}^{k_0+N_p-1}x^2\left(k\right)&&\text{Periodische Signale}\\
&=X_0^2&&\text{Konstantes Signale}
\end{align*}
\end{boxrightshaded}

\begin{boxleft}\bla{Effektivwert}
\des{x_{eff}}{Effektivwert}
\end{boxleft}\begin{boxrightshaded}
\begin{align*}
x_{eff}&=\sqrt{P_x}=\sqrt{\lim_{T\to \infty}\int_{t_0}^{t_0+T} x^2\left(t\right)\diff t}\\
&=\sqrt{\frac{1}{n\cdot T_p}\int_{t_0}^{t_0+n\cdot T_p} x^2\left(t\right)\diff t}&&\text{Periodische Signale}\\
&=\sqrt{\lim_{N\to \infty}\frac{1}{N}\sum_{k=k_0}^{k_0+N-1}x^2\left(k\right)}\\
&=\sqrt{\frac{1}{N_p}\sum_{k=k_0}^{k_0+N_p-1}x^2\left(k\right)}&&\text{Periodische Signale}\\
&=X_0&&\text{Konstantes Signale}
\end{align*}
\end{boxrightshaded}


\begin{boxleft}\bla{Gleichanteil}
\des{\bar{x}}{Gleichanteil}\\
\des{m_1}{gewöhnliches Moment 1. Ordnung}\\
\des{E\left(x\right)}{Erwartungswert}
\end{boxleft}\begin{boxrightshaded}
\begin{align*}
\bar{x}&=m_1=E\left(x\right)=\lim_{T \to \infty}\int_{t_0}^{t_0+T}x\left(t\right) \diff t\\
&=\frac{1}{n\cdot T_p}\int_{t_0}^{t_0+n\cdot T_p}u\left(t\right)\diff t&&\text{Periodische Signale}\\
&=\lim_{N\to\infty}\frac{1}{N}\sum_{k=k_0}^{k_0+N-1}x\left(t\right)\\
&=\frac{1}{N_p}\sum_{k=k_0}^{k_0+N_p-1}x\left(t\right)&&\text{Periodische Signale}\\
&=X_0&&\text{Konstantes Signale}
\end{align*}
\end{boxrightshaded}


\begin{boxleft}\bla{Signalgleichleistung}
\des{P_{x=}}{Signalgleichleistung}\\
\des{\bar{x}^2}{Quadratisch linearer Mittelwert}\\
\des{m_1^2}{Quadratisch g. Moment 1. Ordnung}
\end{boxleft}\begin{boxrightshaded}
\begin{align*}
P_{x=}&=\left(\bar{x}\right)^2=m_1^2=\left(\lim_{T \to \infty}\int_{t_0}^{t_0+T}x\left(t\right) \diff t\right)^2\\
&=\left(\frac{1}{n\cdot T_p}\int_{t_0}^{t_0+n\cdot T_p}u\left(t\right)\diff t\right)^2&&\text{Periodische Signale}\\
&=\left(\lim_{N\to\infty}\frac{1}{N}\sum_{k=k_0}^{k_0+N-1}x\left(t\right)\right)^2\\
&=\left(\frac{1}{N_p}\sum_{k=k_0}^{k_0+N_p-1}x\left(t\right)\right)^2&&\text{Periodische Signale}\\
&=\left(X_0\right)^2&&\text{Konstantes Signale}
\end{align*}
\end{boxrightshaded}


\begin{boxleft}\bla{Signalwechselleistung}
\des{P_{x~}}{Signalwechselleistung}\\
\des{\sigma^2}{Varianz}\\
\des{\mu_2}{Quadratisch g. Moment 2. Ordnung}
\end{boxleft}\begin{boxrightshaded}
\begin{align*}
P_{x~}&=\sigma^2=\mu_2=\bar{x^2}-\bar{x}^2\\
&=\lim_{T\to\infty}\frac{1}{T}\int_{t_0}^{t_0+T}\left(x\left(t\right)-\bar{x}\right)^2\diff t\\
&=\frac{1}{n\cdot T_p}\int_{t_0}^{t_0+n\cdot T_p}\left(x\left(t\right)-\bar{x}\right)^2\diff t&&\text{Periodische Signale}\\
&=\lim_{N\to\infty}\frac{1}{N}\sum_{k=k_0}^{k_0+N-1}\left(x\left(k\right)-\bar{x}\right)^2\\
&=\frac{1}{N_p}\sum_{k=k_0}^{k_0+N_p-1}\left(x\left(k\right)-\bar{x}\right)^2&&\text{Periodische Signale}\\
&=0&&\text{Konstantes Signale}
\end{align*}
\end{boxrightshaded}

\begin{boxleft}\bla{Standartabweichung}
\des{\sigma}{Standartabweichung}
\end{boxleft}\begin{boxrightshaded}
\begin{align*}
\sigma=\sqrt{P_{x~}}&=\sqrt{\mu_2}
\end{align*}
\end{boxrightshaded}


\section{Systeme}

\subsection{Linearität}

\begin{boxleft}\bla{Homogenität}
\end{boxleft}\begin{boxrightshaded}
\begin{align*}
x\left(t\right)&=C\cdot x_1\left(t\right)\to y\left(t\right)=C\cdot y_1\left(t\right)
\end{align*}
\end{boxrightshaded}

\begin{boxleft}\bla{Additivität}
\end{boxleft}\begin{boxrightshaded}
\begin{align*}
x\left(t\right)&= x_1\left(t\right)+x_2\left(t\right)\to y\left(t\right)= y_1\left(t\right)+y_2\left(t\right)
\end{align*}
\end{boxrightshaded}


\subsection{Zeitinvarianz}

\begin{boxleft}\bla{Zeitinvarianz}
\end{boxleft}\begin{boxrightshaded}
\begin{align*}
x\left(t\right)&=x_1\left(t-\tau\right)\to y\left(t\right)=y_1\left(t-\tau\right)
\end{align*}
\end{boxrightshaded}


\subsection{Kausalität}

\begin{boxleft}\bla{Zeitinvarianz}
\end{boxleft}\begin{boxrightshaded}
\begin{align*}
x\left(t\right)&=x_1\left(t-\tau\right)\to y\left(t\right)=y_1\left(t-\tau\right)
\end{align*}
\end{boxrightshaded}
