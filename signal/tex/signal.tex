\chapter{Signal- und Systemtheorie}

\section{Grundsignale}

\subsection{Einheitsignale}
\begin{boxleft}\bla{Diracstoß}
\des[\per\second]{\delta\left(t\right)}{Diracstoß}
\end{boxleft}\begin{boxrightshaded}
\begin{align*}
\delta\left(t\right)&=
\begin{dcases*}
  0\si{\per\second}&$t<0$\\
\infty\si{\per\second}& $t=0$\\
  0\si{\per\second}&$t>0$
\end{dcases*}\\
\int_{-\infty}^\infty\delta\left(t\right)\diff t&=1\\
\delta\left(t\right)&=\frac{\diff \sigma\left(t\right)}{\diff t}=\frac{\diff^2 \alpha\left(t\right)}{\diff t^2}
\end{align*}
\end{boxrightshaded}


\begin{boxleft}\bla{Einheitssprungsfunktion}
\des{\sigma\left(t\right)}{Einheitssprungsfunktion}
\end{boxleft}\begin{boxrightshaded}
\begin{align*}
\sigma\left(t\right)&=
\begin{dcases*}
  0&$t<0$\\
0,5& $t=0$\\
  1&$t>0$
\end{dcases*}\\
\sigma\left(t\right)&=\int_{-\infty}^t \delta\left(t\right)\diff t=\frac{\diff \alpha\left(t\right)}{\diff t}
\end{align*}
\end{boxrightshaded}


\begin{boxleft}\bla{Einheitsanstiegsfunktion}
\des[\second]{\alpha\left(t\right)}{Einheitsanstiegsfunktion}
\end{boxleft}\begin{boxrightshaded}
\begin{align*}
\alpha\left(t\right)&=
\begin{dcases*}
  0\si{\second}&$t<0$\\
t& $t=0$
\end{dcases*}\\
\alpha\left(t\right)&=\iint_{-\infty}^t \delta\left(t\right)\diff t\diff t=\int_{-\infty}^t \sigma\left(t\right)\diff t
\end{align*}
\end{boxrightshaded}


\subsection{Weitere Grundsignale}
\begin{boxleft}\bla{Rechtecksimpuls}
\des{\operatorname{rect}_T\left(t\right)}{Rechtecksimpuls}
\end{boxleft}\begin{boxrightshaded}
\begin{align*}
\operatorname{rect}_T\left(t\right)&=
\begin{dcases*}
  1&$\left|t\right|<\frac{T}{2}$\\
0,5& $\left|t\right|=\frac{T}{2}$\\
0& $\left|t\right|>\frac{T}{2}$
\end{dcases*}
\end{align*}
\end{boxrightshaded}


\begin{boxleft}\bla{Dreiecksimpuls}
\des{\operatorname{\Lambda}_T\left(t\right)}{Dreiecksimpuls}
\end{boxleft}\begin{boxrightshaded}
\begin{align*}
\operatorname{\Lambda}_T\left(t\right)&=
\begin{dcases*}
  1+\frac{t}{T}&$-T<t<0$\\
  1-\frac{t}{T}&$0\leq t<T$\\
  0&$\left|t\right|>T$\\
\end{dcases*}
\end{align*}
\end{boxrightshaded}

\subsection{Signalveränderungen}
\begin{boxleft}\bla{Offset}
\des{X_{off}}{Offsetwert}
\end{boxleft}\begin{boxrightshaded}
\begin{align*}
x_2\left(t\right)&=x_1\left(t\right)+X_{off}
\end{align*}
\end{boxrightshaded}


\begin{boxleft}\bla{Skalierung}
\des{V}{Verstärkungsfaktor}
\end{boxleft}\begin{boxrightshaded}
\begin{align*}
x_2\left(t\right)&=V\cdot x_1\left(t\right)
\end{align*}
\end{boxrightshaded}


\begin{boxleft}\bla{Verschiebung}
\des{t_0}{Verschiebungskonstante}
\end{boxleft}\begin{boxrightshaded}
\begin{align*}
x_2\left(t\right)&= x_1\left(t-t_0\right)&&\text{$t_0>0$: Rechtsverschiebung}
\end{align*}
\end{boxrightshaded}

\begin{boxleft}\bla{Negation des Argumentes}
\end{boxleft}\begin{boxrightshaded}
\begin{align*}
x_2\left(t\right)&= x_1\left(-t\right)&&\text{Spiegelung an der Ordinate}
\end{align*}
\end{boxrightshaded}


\begin{boxleft}\bla{Negiertes und verschobenes Argument}
\des{t_0}{Verschiebungskonstante}
\end{boxleft}\begin{boxrightshaded}
\begin{align*}
x_2\left(t\right)&= x_1\left(-\left(t-t_0\right)\right)&&\text{Spiegelung bei $\frac{t_0}{2}$}
\end{align*}
\end{boxrightshaded}


\begin{boxleft}\bla{Argumentskalierung}
\end{boxleft}\begin{boxrightshaded}
\begin{align*}
x_2\left(t\right)&= x_1\left(a\cdot t\right)&&\text{$a<1$ Streckung der Funktion}
\end{align*}
\end{boxrightshaded}



\section{Signaleigenschaften}

\subsection{Energiesignale}

\begin{boxshaded}
\begin{align*}
&\text{$E=$endlich positiver Wert. $P=0$} 
\end{align*}
\end{boxshaded}

\begin{boxleft}\bla{Energie}
\des[\watt\second]{E_R}{Energie}\\
\des{E_X}{Normierte Signalenergie}
\end{boxleft}\begin{boxrightshaded}
\begin{align*}
E_R&=\int_{-\infty}^\infty u\left(t\right)\cdot i\left(t\right)\diff t\\
&=\int_{-\infty}^\infty \frac{u^2\left(t\right)}{R}\diff t\\
E_x&=m_{i2}=\int_{-\infty}^\infty x^2\left(t\right)\diff t&&\text{Normierung auf $R=1$}\\
E_x&=\sum_{k=-\infty}^{\infty}x^2\left(k\right)
\end{align*}
\end{boxrightshaded}

\begin{boxleft}\bla{Impulsfläsche}
\des[\watt\second]{E_R}{Energie}
\end{boxleft}\begin{boxrightshaded}
\begin{align*}
A_x&=m_{i1}=\int_{-\infty}^\infty x\left(t\right)\diff t\\
A_x&=\sum_{k=-\infty}^{\infty}x\left(k\right)
\end{align*}
\end{boxrightshaded}


\subsection{Leistungssignale}


\begin{boxshaded}
\begin{align*}
&\text{$E=\infty$. $P=$endlich positiver Wert.} 
\end{align*}
\end{boxshaded}

\begin{boxleft}\bla{Mittlere Signalleistung}
\des{P_x}{Mittlere Signalleistung}\\
\des{\bar{x^2}}{quadratischer Mittelwert}\\
\des{m_2}{gewöhnliches Moment 2. Ordnung}\\
\des{x_0^2}{Konstantes Signale}
\end{boxleft}\begin{boxrightshaded}
\begin{align*}
P_x&=\bar{x^2}=m_2=\lim_{T\to \infty}\int_{t_0}^{t_0+T} x^2\left(t\right)\diff t\\
&=\frac{1}{n\cdot T_p}\int_{t_0}^{t_0+n\cdot T_p} x^2\left(t\right)\diff t&&\text{Periodische Signale}\\
&=\lim_{N\to \infty}\frac{1}{N}\sum_{k=k_0}^{k_0+N-1}x^2\left(k\right)\\
&=\frac{1}{N_p}\sum_{k=k_0}^{k_0+N_p-1}x^2\left(k\right)&&\text{Periodische Signale}\\
&=X_0^2&&\text{Konstantes Signale}
\end{align*}
\end{boxrightshaded}

\begin{boxleft}\bla{Effektivwert}
\des{x_{eff}}{Effektivwert}
\end{boxleft}\begin{boxrightshaded}
\begin{align*}
x_{eff}&=\sqrt{P_x}=\sqrt{\lim_{T\to \infty}\int_{t_0}^{t_0+T} x^2\left(t\right)\diff t}\\
&=\sqrt{\frac{1}{n\cdot T_p}\int_{t_0}^{t_0+n\cdot T_p} x^2\left(t\right)\diff t}&&\text{Periodische Signale}\\
&=\sqrt{\lim_{N\to \infty}\frac{1}{N}\sum_{k=k_0}^{k_0+N-1}x^2\left(k\right)}\\
&=\sqrt{\frac{1}{N_p}\sum_{k=k_0}^{k_0+N_p-1}x^2\left(k\right)}&&\text{Periodische Signale}\\
&=X_0&&\text{Konstantes Signale}
\end{align*}
\end{boxrightshaded}


\begin{boxleft}\bla{Gleichanteil}
\des{\bar{x}}{Gleichanteil}\\
\des{m_1}{gewöhnliches Moment 1. Ordnung}\\
\des{E\left(x\right)}{Erwartungswert}
\end{boxleft}\begin{boxrightshaded}
\begin{align*}
\bar{x}&=m_1=E\left(x\right)=\lim_{T \to \infty}\int_{t_0}^{t_0+T}x\left(t\right) \diff t\\
&=\frac{1}{n\cdot T_p}\int_{t_0}^{t_0+n\cdot T_p}u\left(t\right)\diff t&&\text{Periodische Signale}\\
&=\lim_{N\to\infty}\frac{1}{N}\sum_{k=k_0}^{k_0+N-1}x\left(k\right)\\
&=\frac{1}{N_p}\sum_{k=k_0}^{k_0+N_p-1}x\left(k\right)&&\text{Periodische Signale}\\
&=X_0&&\text{Konstantes Signale}
\end{align*}
\end{boxrightshaded}


\begin{boxleft}\bla{Signalgleichleistung}
\des{P_{x=}}{Signalgleichleistung}\\
\des{\bar{x}^2}{Quadratisch linearer Mittelwert}\\
\des{m_1^2}{Quadratisch g. Moment 1. Ordnung}
\end{boxleft}\begin{boxrightshaded}
\begin{align*}
P_{x=}&=\left(\bar{x}\right)^2=m_1^2=\left(\lim_{T \to \infty}\int_{t_0}^{t_0+T}x\left(t\right) \diff t\right)^2\\
&=\left(\frac{1}{n\cdot T_p}\int_{t_0}^{t_0+n\cdot T_p}u\left(t\right)\diff t\right)^2&&\text{Periodische Signale}\\
&=\left(\lim_{N\to\infty}\frac{1}{N}\sum_{k=k_0}^{k_0+N-1}x\left(k\right)\right)^2\\
&=\left(\frac{1}{N_p}\sum_{k=k_0}^{k_0+N_p-1}x\left(k\right)\right)^2&&\text{Periodische Signale}\\
&=\left(X_0\right)^2&&\text{Konstantes Signale}
\end{align*}
\end{boxrightshaded}


\begin{boxleft}\bla{Signalwechselleistung}
\des{P_{x~}}{Signalwechselleistung}\\
\des{\sigma^2}{Varianz}\\
\des{\mu_2}{Quadratisch g. Moment 2. Ordnung}
\end{boxleft}\begin{boxrightshaded}
\begin{align*}
P_{x~}&=\sigma^2=\mu_2=\bar{x^2}-\bar{x}^2\\
&=\lim_{T\to\infty}\frac{1}{T}\int_{t_0}^{t_0+T}\left(x\left(t\right)-\bar{x}\right)^2\diff t\\
&=\frac{1}{n\cdot T_p}\int_{t_0}^{t_0+n\cdot T_p}\left(x\left(t\right)-\bar{x}\right)^2\diff t&&\text{Periodische Signale}\\
&=\lim_{N\to\infty}\frac{1}{N}\sum_{k=k_0}^{k_0+N-1}\left(x\left(k\right)-\bar{x}\right)^2\\
&=\frac{1}{N_p}\sum_{k=k_0}^{k_0+N_p-1}\left(x\left(k\right)-\bar{x}\right)^2&&\text{Periodische Signale}\\
&=0&&\text{Konstantes Signale}
\end{align*}
\end{boxrightshaded}

\begin{boxleft}\bla{Standartabweichung}
\des{\sigma}{Standartabweichung}
\end{boxleft}\begin{boxrightshaded}
\begin{align*}
\sigma=\sqrt{P_{x~}}&=\sqrt{\mu_2}
\end{align*}
\end{boxrightshaded}


\section{Systeme}

\subsection{Linearität}

\begin{boxleft}\bla{Homogenität}
\end{boxleft}\begin{boxrightshaded}
\begin{align*}
x\left(t\right)&=C\cdot x_1\left(t\right)\Rightarrow y\left(t\right)=C\cdot y_1\left(t\right)
\end{align*}
\end{boxrightshaded}

\begin{boxleft}\bla{Additivität}
\end{boxleft}\begin{boxrightshaded}
\begin{align*}
x\left(t\right)&= x_1\left(t\right)+x_2\left(t\right)\Rightarrow y\left(t\right)= y_1\left(t\right)+y_2\left(t\right)
\end{align*}
\end{boxrightshaded}


\subsection{Zeitinvarianz}

\begin{boxshaded}
\begin{align*}
\text{Zeitinvariante Systeme ändern ihre Eigenschaften nicht mit der Zeit.}
\end{align*}
\end{boxshaded}

\begin{boxleft}\bla{Zeitinvarianz}
\end{boxleft}\begin{boxrightshaded}
\begin{align*}
x\left(t\right)&=x_1\left(t-\tau\right)\Rightarrow y\left(t\right)=y_1\left(t-\tau\right)
\end{align*}
\end{boxrightshaded}


\subsection{Kausalität}

\begin{boxshaded}
\begin{align*}
\text{Bei Kausalen Systemen gibt es kein Ereigniss am Ausgang ohne ein entsprechendes Eingangssignal.}
\end{align*}
\end{boxshaded}

\begin{boxleft}\bla{Kausalität}
\end{boxleft}\begin{boxrightshaded}
\begin{align*}
\left.\frac{\diff x\left(t\right)}{\diff t}\right|_{t<t_0}\Rightarrow\left.\frac{\diff y\left(t\right)}{\diff t}\right|_{t<t_0}
\end{align*}
\end{boxrightshaded}

\subsection{Stabilität}


\begin{boxshaded}
\begin{align*}
\text{Stabilität ist ein System wenn es auf eine begrenztes Eingangssignal, nicht mit einen unendlichen Ausgangssignal reagiert.}
\end{align*}
\end{boxshaded}

\begin{boxleft}\bla{Stabil}
\end{boxleft}\begin{boxrightshaded}
\begin{align*}
\int_{-\infty}^{\infty}x\left(t\right)\diff t<\infty\Rightarrow\int_{-\infty}^{\infty}y\left(t\right)\diff t<\infty
\end{align*}
\end{boxrightshaded}


\begin{boxleft}\bla{Grenzstabil}
\end{boxleft}\begin{boxrightshaded}
\begin{align*}
\int_{-\infty}^{\infty}x\left(t\right)\diff t<\infty\Rightarrow\left.\lim_{T\to \infty}\int_{t_0}^{t_0+T} y^2\left(t\right)\diff t\right|_{t>\tau}=\text{const}
\end{align*}
\end{boxrightshaded}

\subsection{Umwandung unterschiedlicher Eingangssignalen}


\begin{boxleft}\bla{Diracstoß-Eingangssignal}
\end{boxleft}\begin{boxrightshaded}
\begin{align*}
x\left(t\right)&=\delta\left(t\right)\Rightarrow y\left(t\right)&=g\left(t\right)
\end{align*}
\end{boxrightshaded}

\section{Signalverarbeitung}

\subsection{Zerlegung Gerade u. Ungerade}


\begin{boxleft}\bla{Gerade u. Ungerade}
\end{boxleft}\begin{boxrightshaded}
\begin{align*}
x\left(t\right)&=x_g\left(t\right)+x_u\left(t\right)\\
x_g\left(t\right)&=\frac{x\left(t\right)+x\left(-t\right)}{2}\\
x_u\left(t\right)&=\frac{x\left(t\right)-x\left(-t\right)}{2}
\end{align*}
\end{boxrightshaded}


\subsection{Faltung}


\begin{boxshaded}
\begin{align*}
\text{Faltung entspricht graphisch eine Spiegelung eines Signals und dessen Verschiebung über einem anderen Signal.}
\end{align*}
\end{boxshaded}

\begin{boxleft}\bla{Faltungsintegral}
\end{boxleft}\begin{boxrightshaded}
\begin{align*}
y\left(t\right)&=x_1\left(t\right)\ast x_2\left(t\right)\\
y\left(t\right)&=\int_{-\infty}^{\infty}x_1\left(\tau\right)x_2\left(t-\tau\right)\diff \tau
\end{align*}
\end{boxrightshaded}


\subsection{Laplace-Transformation}

\begin{boxleft}\bla{Laplaceintegral}
\end{boxleft}\begin{boxrightshaded}
\begin{align*}
X\left(p\right)&=\mathscr{L}\{x\left(t\right)\}=\int_{0}^{\infty}x\left(t\right)e^{-p\cdot t}\diff t\\
X\left(p\right)&\quad\Laplace\quad x\left(t\right)
\end{align*}
\end{boxrightshaded}


\subsection{Fourier-Transformation}

\begin{boxleft}\bla{Fouriersintegral}
\end{boxleft}\begin{boxrightshaded}
\begin{align*}
X\left(\omega\right)&=\mathscr{F}\{x\left(t\right)\}=\int_{-\infty}^{\infty}x\left(t\right)e^{-j\omega t}\diff t\\
X\left(\omega\right)&\quad\Laplace\quad x\left(t\right)\\
X\left(f\right)&=\mathscr{F}\{x\left(t\right)\}=\int_{-\infty}^{\infty}x\left(t\right)e^{-j2\pi f t}\diff t\\
X\left(f\right)&=\int_{-\infty}^\infty\left(x_{re}+jx_{im}\right)\cdot\left(\cos\left(2\pi f t\right)-j\cdot\sin\left(2\pi f t\right)\right)\diff t\\
X\left(f\right)&\quad\Laplace\quad x\left(t\right)\\
x\left(t\right)&=\frac{1}{2\cdot\pi}\cdot\int_{-\infty}^{\infty}X\left(\omega\right)e^{j\omega t}\diff \omega\\
x\left(t\right)&=\int_{-\infty}^{\infty}X\left(f \right)e^{j2\pi f t}\diff \omega
\end{align*}
\end{boxrightshaded}

\begin{boxleft}\bla{Additionssatz}
\end{boxleft}\begin{boxrightshaded}
\begin{align*}
x\left(t\right)&=x_1\left(t\right)+x_2\left(t\right)+\dots\quad\Laplace\quad X\left(f\right)=X_1\left(f\right)+X_2\left(f\right)+\dots
\end{align*}
\end{boxrightshaded}

\begin{boxleft}\bla{Linearität}
\end{boxleft}\begin{boxrightshaded}
\begin{align*}
x\left(t\right)&=C\cdot x_1\left(t\right)\quad&\Laplace\quad X\left(f\right)&=C\cdot X_1\left(f\right)
\end{align*}
\end{boxrightshaded}

\begin{boxleft}\bla{Verschiebungssatz}
\end{boxleft}\begin{boxrightshaded}
\begin{align*}
x\left(t\right)&=x_1\left(t-t_0\right)\quad&\Laplace\quad X\left(f\right)&=X_1\left(f\right)\cdot e^{-j2\pi f\cdot t_0}
\end{align*}
\end{boxrightshaded}

\begin{boxleft}\bla{Ähnlichkeitssatz}
\end{boxleft}\begin{boxrightshaded}
\begin{align*}
x\left(t\right)&=x_1\left(a\cdot t\right)\quad&\Laplace\quad X\left(f\right)&=\frac{1}{\left|a\right|}X_1\left(\frac{f}{a}\right)\\
x\left(t\right)&=\frac{1}{\left|b\right|}x_1\left(\frac{t}{b}\right)\quad&\Laplace\quad X\left(f\right)&=X_1\left(b\cdot f\right)
\end{align*}
\end{boxrightshaded}

\begin{boxleft}\bla{Differentationssatz}
\end{boxleft}\begin{boxrightshaded}
\begin{align*}
x\left(t\right)&=\frac{\diff x_1\left(t\right)}{\diff t}\quad&\Laplace\quad X\left(f\right)&=j2\pi f\cdot X\left(f\right)\\
x\left(t\right)&=\frac{\diff^K x_1\left(t\right)}{\diff t^K}\quad&\Laplace\quad X\left(f\right)&=j^K\left(2\pi f\right)^K\cdot X\left(f\right)\\
x\left(t\right)&=\frac{\diff^K x_1\left(t\right)}{\diff t^K}\quad&\Laplace\quad X\left(\omega\right)&=j^K\left(\omega\right)^K\cdot X\left(\omega\right)
\end{align*}
\end{boxrightshaded}

\begin{boxleft}\bla{Integrationssatz}
\end{boxleft}\begin{boxrightshaded}
\begin{align*}
x\left(t\right)&=\int_\infty^tx_1\left(\tau\right)\diff\tau\quad&\Laplace\quad X\left(f\right)&=\frac{1}{j2\pi f}\cdot X_1\left(f\right)+\frac{1}{2}X_1\left(f=0\right)\delta\left(f\right)\\
x\left(t\right)&=\int_\infty^tx_1\left(\tau\right)\diff\tau\quad&\Laplace\quad X\left(\omega\right)&=\frac{1}{j \omega}\cdot X_1\left(\omega\right)+\pi\cdot X_1\left(\omega=0\right)\delta\left(\omega\right)
\end{align*}
\end{boxrightshaded}

\begin{boxleft}\bla{Integrationssatz im Frequenzbereich}
\end{boxleft}\begin{boxrightshaded}
\begin{align*}
x\left(t\right)&=\frac{1}{-j2\pi t}\cdot x_1\left(t\right)+\frac{1}{2}x_1\left(t=0\right)\delta\left(t\right)\quad&\Laplace\quad X\left(f\right)&=\int_{-\infty}^{f}X_1\left(\varphi\right)\diff \varphi\\
x\left(t\right)&=\frac{1}{-j t}\cdot x_1\left(t\right)+\pi\cdot x_1\left(t=0\right)\delta\left(t\right)\quad&\Laplace\quad X\left(\omega\right)&=\int_{-\infty}^{\omega}X_1\left(\varphi\right)\diff \varphi
\end{align*}
\end{boxrightshaded}

\begin{boxleft}\bla{Vertauschungssatz}
\end{boxleft}\begin{boxrightshaded}
\begin{align*}
x\left(t\right)&=x_1\left(t\right)\quad&\Laplace\quad X\left(f\right)&=X_1\left(f\right)\\
x\left(t\right)&=X_1\left(t\right)\quad&\Laplace\quad X\left(f\right)&=x_1\left(-f\right)\\
x\left(t\right)&=x_1\left(t\right)\quad&\Laplace\quad X\left(\omega\right)&=X_1\left(\omega\right)\\
x\left(t\right)&=X_1\left(t\right)\quad&\Laplace\quad X\left(\omega\right)&=2\pi\cdot x_1\left(-\omega\right)
\end{align*}
\end{boxrightshaded}

\begin{boxleft}\bla{Faltung}
\end{boxleft}\begin{boxrightshaded}
\begin{align*}
x\left(t\right)&=x_1\left(t\right)\quad&\Laplace\quad X\left(f\right)&=\int_{-\infty}^\infty X_1\left(\varphi\right)\cdot X_2\left(f-\varphi\right)\diff \varphi\\
x\left(t\right)&=x_1\left(t\right)\quad&\Laplace\quad X\left(\omega\right)&=\frac{1}{2\pi}\int_{-\infty}^\infty X_1\left(\varphi\right)\cdot X_2\left(\omega-\varphi\right)\diff \varphi\\
x\left(t\right)&=\int_{-\infty}^\infty x_1\left(\tau\right)\cdot x_2\left(t-\tau\right)\diff \tau \quad&\Laplace\quad X\left(f\right)&=x_1\left(f\right)\cdot x_2\left(f\right) 
\end{align*}
\end{boxrightshaded}

\begin{boxleft}\bla{Delta-Impulsfläsche}
\end{boxleft}\begin{boxrightshaded}
\begin{align*}
\sha_p\left(t\right)&=\sum_{k=-\infty}^{\infty}\delta\left(t-kt_p\right)\quad&\Laplace\quad\Sha_A\left(f\right)&=f_A\sum_{m=-\infty}^{\infty}\delta\left(f-mf_a\right)\\
f_a&=\frac{1}{t_p}\\
\sha_a\left(t\right)&=\sum_{k=-\infty}^{\infty}t_a\delta\left(t-kt_a\right)\quad&\Laplace\quad\Sha_P\left(f\right)&=\sum_{m=-\infty}^{\infty}\delta\left(f-mf_p\right)\\
f_p&=\frac{1}{t_a}
\end{align*}
\end{boxrightshaded}

\begin{boxleft}\bla{Periodifizierung}
\end{boxleft}\begin{boxrightshaded}
\begin{align*}
x\left(t\right)&=x_T\left(t\right)\ast\sha_p\left(t\right)\quad&\Laplace\quad X\left(f\right)&=X_T\left(f\right)\cdot \Sha_A\left(f\right)
\end{align*}
\end{boxrightshaded}

\begin{boxleft}\bla{Abgetastete Funktionen}
\end{boxleft}\begin{boxrightshaded}
\begin{align*}
x_\delta\left(t\right)&=x\left(t\right)\cdot\sha_a\left(t\right)\quad&\Laplace\quad X_\delta\left(f\right)&=X\left(f\right)\ast\Sha_p\left(f\right)\\
x_\delta\left(t\right)&=\sum_{k=-\infty}^{\infty}x\left(kt_a\right)\cdot t_a\cdot \delta\left(t-kt_a\right)\quad&\Laplace\quad X_\delta&=\sum_{m=-\infty}^{\infty}X\left(f-mf_p\right)
\end{align*}
\end{boxrightshaded}


\begin{boxleft}\bla{Abgetastete und Periodifizierte Funktionen}
\end{boxleft}\begin{boxrightshaded}
\begin{align*}
x_{\delta p}\left(t\right)&=\left(x_T\left(t\right)\ast\sha_p\left(t\right)\right)\cdot\sha_a\left(t\right) \\
x_{\delta p}\left(t\right)&=\sum_{m=-\infty}^\infty\sum_{k=-\infty}^\infty x_T\left(kt_a-mt_p\right)\cdot t_a\cdot \delta\left(t-kt_a\right)\\
X_{\delta p}\left(t\right)&=\left(X_T\left(f\right)\cdot \Sha_a\left(f\right)\right)\ast\Sha_p\left(f\right)\\
X_{\delta p}\left(t\right)&=\sum_{m=-\infty}^\infty\sum_{k=-\infty}^\infty X_T\left(mf_a-kf_p\right)\cdot f_a\cdot \delta\left(f-mf_a\right)\\
f_a&=\frac{1}{t_p}&f_p&=\frac{1}{t_a}
\end{align*}
\end{boxrightshaded}

\begin{boxleft}\bla{Korrespodenz}
\end{boxleft}\begin{boxrightshaded}
\begin{align*}
x\left(t\right)&=\hat{X}\mathrm{rect}_{T}\left(t\right)\quad&\laplace\quad X\left(f\right)&=\hat{X}T\cdot\mathrm{si}\left(\pi\cdot f\cdot T\right)\\
x\left(t\right)&=\hat{X}\Lambda_{T}\left(t\right)\quad&\laplace\quad X\left(f\right)&=\hat{X}T\cdot\mathrm{si}^2\left(\pi\cdot f\cdot T\right)\\
x\left(t\right)&=\hat{X}\sin\left(2\pi f_0 t\right)\quad&\laplace\quad X\left(f\right)&=\frac{j\hat{X}}{2}\left(\delta\left(f+f_0\right)-\delta\left(f-f_0\right)\right)\\
x\left(t\right)&=\hat{X}\cos\left(2\pi f_0 t\right)\quad&\laplace\quad X\left(f\right)&=\frac{\hat{X}}{2}\left(\delta\left(f+f_0\right)+\delta\left(f-f_0\right)\right)
\end{align*}
\end{boxrightshaded}

\subsection{DFT und FFT}

\begin{boxleft}\bla{DFT}
\end{boxleft}\begin{boxrightshaded}
\begin{align*}
\text{1. Variante}\\
X\left(l\right)&=\sum_{k=0}^{N-1}x\left(k\right)e^{-j2\pi\cdot l \cdot\frac{k}{N}}\\
x\left(k\right)&=\frac{1}{N}\sum_{k=0}^{N-1}x\left(k\right)e^{j2\pi\cdot l \cdot\frac{k}{N}}\\
\text{2. Variante}\\
X\left(l\right)&=\frac{1}{N}\sum_{k=0}^{N-1}x\left(k\right)e^{-j2\pi\cdot l \cdot\frac{k}{N}}\\
x\left(k\right)&=\sum_{k=0}^{N-1}x\left(k\right)e^{j2\pi\cdot l \cdot\frac{k}{N}}\\
\text{3. Variante}\\
X\left(l\right)&=\frac{1}{\sqrt{N}}\sum_{k=0}^{N-1}x\left(k\right)e^{-j2\pi\cdot l \cdot\frac{k}{N}}\\
x\left(k\right)&=\frac{1}{\sqrt{N}}\sum_{k=0}^{N-1}x\left(k\right)e^{j2\pi\cdot l \cdot\frac{k}{N}}
\end{align*}
\end{boxrightshaded}

\begin{boxleft}\bla{DFT als Matrix-Multiplikation}
\end{boxleft}\begin{boxrightshaded}
\begin{align*}
\left[X\left(l\right)\right]&=\left[F_{l,k}^N\right]\cdot\left[x\left(k\right)\right]&t&\Rightarrow f\\
\left[x\left(k\right)\right]&=\left[f_{k,l}^N\right]\cdot\left[X\left(l\right)\right]&f&\Rightarrow t\\
\left[f_{k,l}^N\right]&=\left[F_{l,k}^N\right]^*\\
F_{l,k}^N&=e^{-j\pi \cdot l\cdot \frac{k}{N}}&\Rightarrow F_{l,k}^N&=\cos\left(2\pi\cdot \frac{l\cdot k}{N}\right)-j\sin\left(2\pi\cdot \frac{l\cdot k}{N}\right)
\end{align*}
\end{boxrightshaded}


\subsection{Spektrum}

\begin{boxleft}\bla{Betragsspektrum}
\end{boxleft}\begin{boxrightshaded}
\begin{align*}
\left|X\left(f\right)\right|&=\sqrt{\left(\text{Re}\left\{X\left(f\right)\right\}\right)^2+\left(\text{Im}\left\{X\left(f\right)\right\}\right)^2}
\end{align*}
\end{boxrightshaded}

\begin{boxleft}\bla{Betragsquadratspektrum}
\end{boxleft}\begin{boxrightshaded}
\begin{align*}
\left|X\left(f\right)\right|^2&=\left(\text{Re}\left\{X\left(f\right)\right\}\right)^2+\left(\text{Im}\left\{X\left(f\right)\right\}\right)^2
\end{align*}
\end{boxrightshaded}

\begin{boxleft}\bla{Theorem von Parseval}
\end{boxleft}\begin{boxrightshaded}
\begin{align*}
E&=m_{i2}=\int_{-\infty}^\infty x^2\left(t\right)\diff t=\int_{-\infty}^\infty \left|X\left(f\right)\right|^2\diff f
\end{align*}
\end{boxrightshaded}

\subsection{Korrelation}

\begin{boxleft}\bla{Kreuzkorrelationsfunktion}
\end{boxleft}\begin{boxrightshaded}
\begin{align*}
E_{x_1x_2}\left(\tau\right)&=\int_{-\infty}^\infty x_2\left(t+\tau\right)\cdot x_1\left(t\right)\diff t=\int_{-\infty}^\infty x_1\left(t-\tau\right)\cdot x_2\left(t\right)\diff t\\
E_{x_1x_2}\left(l\right)&=\sum_{k=-\infty}^\infty x_2\left(k+l\right)\cdot x_1\left(k\right)\diff t
\end{align*}
\end{boxrightshaded}

\begin{boxleft}\bla{Normierte Kreuzkorrelationsfunktion}
\end{boxleft}\begin{boxrightshaded}
\begin{align*}
\mathring{x}&=\sqrt[n]{x_1\cdot x_2\cdot \dotsc \cdot x_n}\\
\mathring{E}_{x_1x_2}&=\sqrt{\int_{-\infty}^\infty x_1^2\left(t\right)\diff t\cdot \int_{-\infty}^\infty x_2^2\left(t\right)\diff t}\\
r_{x_1x_2}\left(\tau\right)&=\frac{E_{x_1x_2}\left(\tau\right)}{\mathring{E}_{x_1x_2}}=\frac{\int_{-\infty}^\infty x_2\left(t+\tau\right)\cdot x_1\left(t\right)\diff t}{\sqrt{\int_{-\infty}^\infty x_1^2\left(t\right)\diff t\cdot \int_{-\infty}^\infty x_2^2\left(t\right)\diff t}}\\
\left|r_{x_1x_2}\left(\tau\right)\right|&\leq 0
\end{align*}
\end{boxrightshaded}
