 
\chapter{Binäre Rechenoperation}

\section{Zahlensystem}

\begin{boxleft}\bla{Dualsystem}
\end{boxleft}\begin{boxrightshaded}
\begin{align*}
&\text{Basis}:2\quad z\in\left(1;0\right)\\
&\text{Format}:2^{n-1} \dots 2^{0},2^{-1} \dots 2^{-m}\\
&\text{Zahlenwert}:\sum_{l=-m}^{n-1}z_l\cdot2^l
\end{align*}
\end{boxrightshaded}

\begin{boxleft}\bla{Trenärsystem}
\end{boxleft}\begin{boxrightshaded}
\begin{align*}
\text{Basis}:2\quad z\in\left(1;0;-1\right)
\end{align*}
\end{boxrightshaded}

\begin{boxleft}\bla{Oktalsystem}
\end{boxleft}\begin{boxrightshaded}
\begin{align*}
\text{Basis}:8\quad z\in\left(0;1;2;3;4;5;6;7\right)
\end{align*}
\end{boxrightshaded}

\begin{boxleft}\bla{Hexadezimalsystem}
\end{boxleft}\begin{boxrightshaded}
\begin{align*}
\text{Basis}:16\quad z\in\left(0;1;2;3\dots d;e;f\right)
\end{align*}
\end{boxrightshaded}

\begin{boxleft}\bla{Dezimalsystem}
\end{boxleft}\begin{boxrightshaded}
\begin{align*}
\text{Basis}:10\quad z\in\left(0;1;2;3\dots 8;9\right)
\end{align*}
\end{boxrightshaded}

\begin{boxleft}\bla{Stellenberechnung}
\des{k}{Zahlenwert}\\
\des{l}{Anzahl Nachkommastellen in Dezimalsystem}\\
\des{n}{Anzahl Vorkommanstellen in Dualsystem}\\
\des{m}{Anzahl Nachkommastellen in Dualsystem}
\end{boxleft}\begin{boxrightshaded}
\begin{align*}
&n\geq\text{ceil}\left(\frac{\lg\left(k+1\right)}{\lg\left(2\right)}\right)\\
&m\geq\text{ceil}\left(\frac{l}{\lg\left( 2\right)}\right)
\end{align*}
\end{boxrightshaded}

\begin{boxleft}\bla{Wertebereich und Quantisierungsfehler(Dualsystem)}
\des{n}{Anzahl Vorkommanstellen in Dualsystem}\\
\des{m}{Anzahl Nachkommastellen in Dualsystem}
\end{boxleft}\begin{boxrightshaded}
\begin{align*}
&\text{Wertebereich}=0\dots 2^n-2^{-m}\\
&\text{Quantisierungsfehler}=\pm\frac{1}{2}\text{LSB}=\pm 2^{-m-1}\\
&\text{Quantisierungsstuffen}=2^{-m}
\end{align*}
\end{boxrightshaded}

\begin{boxleft}\bla{Umwandlung Negativer Dualzahlen}
\end{boxleft}\begin{boxrightshaded}
\begin{align*}
n&=\text{ceil}\left(1+\frac{\lg\left(|\text{Wert}|\right)}{\lg\left(2\right)}\right)\\
\Delta&=2^{n-1}-\left|\text{Wert}\right|\\
&\text{Umwadlung von $\Delta$ bis MSB Stelle -1, danach setzten der richtigen MSB-Stelle.}
\end{align*}
\end{boxrightshaded}

\section{Addition}

\subsection{Zwei Operanden}

\begin{boxleft}\bla{Halbaddierer}
\des{s_i}{Summe der Stelle i}\\
\des{c_{i+1}}{Übertrag der Stelle i+1}
\end{boxleft}\begin{boxrightshaded}
\begin{align*}
s_i&=a_i\nsim b_i\\
c_{i+1}&=a_i \cdot b_i
\end{align*}
\end{boxrightshaded}

\begin{boxleft}\bla{Volladdierer}
\des{s_i}{Summe der Stelle i}\\
\des{c_{i+1}}{Übertrag der Stelle i+1}\\
\des{c_{i}}{Übertrag der Stelle i}
\end{boxleft}\begin{boxrightshaded}
\begin{align*}
s_i&=a_i\nsim b_i \nsim c_i\\
c_{i+1}&=a_i \cdot b_i+\left(a_i\nsim b_i\right)\cdot c_i
\end{align*}
\end{boxrightshaded}

\begin{boxleft}\bla{Carry-Look-Ahead}
\des{s_i}{Summe der Stelle i}\\
\des{p_{i+1}}{Propagate Übertrag an der Stelle i}\\
\des{g_{i+1}}{Generate Übertrag kompensation}
\end{boxleft}\begin{boxrightshaded}
\begin{align*}
s_i&=a_i\nsim b_i \nsim c_i\\
p_{i+1}&=a_i\nsim b_i\\
g_{i+1}&=a_i\cdot b_i\\
c_{i+1}&=p_{i+1}\cdot c_i +g_{i+1}
\end{align*}
\end{boxrightshaded}

\subsection{Mehrere Operanden}

\begin{boxshaded}
\begin{align*}
&\text{Ripple-Carry: Jede Stuffe addiert jeweils ein Operand hinzu.}\\
&\text{Baumaddierer: Die einzelnen Operaden werden Baumförmig addiert.}\\
&\text{Ripple-Save: Nur der letzte Addierer ist Sequenziel aufgebaut. Daher der Übertrag des Vorgängers wird beim nächsten Aufaddiert.}
\end{align*}
\end{boxshaded}

\subsection{Überlauf}

\begin{boxleft}\bla{Positive Operanden}
\des{s}{Summe}\\
\des{n}{Anzahl Vorkommanstellen in Dualsystem}
\end{boxleft}\begin{boxrightshaded}
\begin{align*}
s&=\left(a+b\right)\text{mod} 2^n
\end{align*}
\end{boxrightshaded}

\begin{boxleft}\bla{Negative Operanden}
\des{s}{Entstehender Summen Wert}\\
\des{p}{Operanden Anzahl}
\end{boxleft}\begin{boxrightshaded}
\begin{align*}
s&=\left(a+b+2^n\text{ceil}\left(\frac{p-1}{2}\right)+2^{n-1}\right)\text{mod} 2^n-2^{n-1}
\end{align*}
\end{boxrightshaded}

\begin{boxleft}\bla{Vermeidung}
\des{r}{Zusätzliche Summenstellen}\\
\des{p}{Operanden Anzahl}
\end{boxleft}\begin{boxrightshaded}
\begin{align*}
r=\text{ceil}\left(\frac{\lg\left(p\right)}{\lg\left(2\right)}\right)
\end{align*}
\end{boxrightshaded}


\subsection{Überlaufserkennung}

\begin{boxleft}\bla{Vergleich MSB Stellen}
\end{boxleft}\begin{boxrightshaded}
\begin{align*}
&\begin{aligned}
&a_{MSB}\neq b_{MSB}\hfill&&\Rightarrow \text{Kein Überlauf möglich}\\
&a_{MSB}=b_{MSB}=s_{MSB}&&\Rightarrow \text{Kein Überlauf möglich}\\
&a_{MSB}=b_{MSB}=0 \quad\&\quad s_{MSB}=1&&\Rightarrow \text{Positiver Überlauf}\\
&a_{MSB}=b_{MSB}=1 \quad\&\quad s_{MSB}=0&&\Rightarrow \text{Negativer Überlauf}
\end{aligned}\\
&\mathit{MIN}=a_{MSB}\\
&\mathit{OVF}=\overline{a}_{MSB}s_{MSB}\left(\overline{b_{MSB}\nsim s_{MSB}}\right)+a_{MSB}\overline{s}_{MSB}\left(b_{MSB}\nsim s_{MSB}\right)
\end{align*}
\end{boxrightshaded}

\begin{boxleft}\bla{Vergleich des Carry}
\end{boxleft}\begin{boxrightshaded}
\begin{align*}
&c_{I;MSB}= c_{O;MSB}&&\Rightarrow \text{Kein Überlauf möglich}\\
&c_{I;MSB}=1\quad\&\quad c_{O;MSB}=0&&\Rightarrow \text{Positiver Überlauf}\\
&c_{I;MSB}=0\quad\&\quad c_{O;MSB}=1&&\Rightarrow \text{Negativer Überlauf}\\
&\mathit{MIN}=c_{O;MSB}\\
&\mathit{OVF}=c_{I;MSB}\nsim c_{O;MSB}
\end{align*}
\end{boxrightshaded}

\begin{boxleft}\bla{Erweiterung der MSB Stelle}
\end{boxleft}\begin{boxrightshaded}
\begin{align*}
&s_{MSB}= s_{MSB+1}&&\Rightarrow \text{Kein Überlauf möglich}\\
&s_{MSB}=1\quad\&\quad s_{MSB+1}=0&&\Rightarrow \text{Positiver Überlauf}\\
&s_{MSB}=0\quad\&\quad s_{MSB+1}=1&&\Rightarrow \text{Negativer Überlauf}\\
&\mathit{MIN}=s_{MSB}\\
&\mathit{OVF}=s_{MSB}\nsim s_{MSB+1}
\end{align*}
\end{boxrightshaded}

\begin{boxleft}\bla{Sättigung}
\des{s_{MSB}}{Behandlung der höchsten Stelle}\\
\des{s_{LSB}}{Behandlung der restlichen Stellen}
\end{boxleft}\begin{boxrightshaded}
\begin{align*}
s'_{MSB}&=s_{MSB}\overline{\mathit{OVF}}+\mathit{OVF}\mathit{MIN}\\
s'_{LSB}&=s_{LSB}\overline{\mathit{OVF}}+\mathit{OVF}\overline{{MIN}}
\end{align*}
\end{boxrightshaded}

\subsection{Umwandlung Tenärcode}

\begin{tabular}{lll|ll}
$h_i$& $b_{i+1}$&$b_i$&$c_i$&$h_{i+1}$\\
\hline
0&0&0&0&0\\
0&0&1&1&0\\
0&1&0&0&0\\
0&1&1&-1&1\\
1&0&0&1&0\\
1&0&1&0&1\\
1&1&0&-1&1\\
1&1&1&0&1
\end{tabular}