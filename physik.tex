\chapter{Physik}
\section{Kinematik}
\subsection{Geradlinige Bewegungen}
\bla{Gesetze}
\begin{shaded}
\begin{align}
	a(t)&=a_0=\frac{\diff v}{\diff t}=\dot{v}=\ddot{s} \\
	v(t)&=a_0*t+v_0=\frac{\diff s}{\diff t}=\dot{s} \\
	s(t)&=\frac{1}{2}a_0*t^2+v_0*t+s_0
\end{align}
\end{shaded}
\subsection{Kreisbewegungen}
 \bla{Gesetze}

\begin{boxleft}
Winkelgrößen
\end{boxleft}\begin{boxrightshaded}
\begin{align}
\alpha(t)&=\alpha_0=\frac{\diff \omega}{\diff t}=\dot{\omega}=\ddot{\varphi} \\
\omega(t)&=\alpha_0*t+\omega_0=\frac{\diff \varphi}{\diff t}=\dot{\varphi} \\
\varphi(t)&=\frac{1}{2}\alpha_0*t^2+\omega_0*t+\varphi_0
\end{align}
\end{boxrightshaded}

\begin{boxleft}
Bahngrößen
\end{boxleft}\begin{boxrightshaded}
\begin{align}
a_t(t)&=a_0=\frac{\diff v}{\diff t}=\dot{v}=\ddot{s} \\
v(t)&=a_0*t+v_0=\frac{\diff s}{\diff t}=\dot{s} \\
s(t)&=\frac{1}{2}a_0*t^2+v_0*t+s_0
\end{align}
\end{boxrightshaded}

\begin{boxleft}Winkelgeschwindigkeit,\\
Kreisfrequenz
\end{boxleft}\begin{boxrightshaded}
\begin{align}
\omega&=\frac{2\cdot\pi}{T}\\
&=2\cdot\pi\cdot n \\
&=2\cdot\pi\cdot f
\end{align}
\end{boxrightshaded}

\begin{boxleft}Bahngeschwindigkeit
\end{boxleft}\begin{boxrightshaded}
\begin{align}
v&=\frac{2\cdot \pi \cdot r}{T}\\
&=\omega\cdot r
\end{align}
\end{boxrightshaded}

\begin{boxleft}Radialbeschleunigung
\end{boxleft}\begin{boxrightshaded}
\begin{align}
a_r&=\frac{v^2}{r}\\
&=v\cdot\omega\\
&=\omega^2\cdot r
\end{align}
\end{boxrightshaded}

\begin{boxleft}Umdrehungen
\end{boxleft}\begin{boxrightshaded}
\begin{align}
N	&=\frac{\omega_0\cdot t}{2\cdot \pi}+\frac{1}{2}\cdot\frac{\alpha}{2\cdot \pi}\cdot t^2\\
	&=n_0\cdot t+\frac{\alpha}{4\cdot\pi}\cdot t^2
\end{align}
\end{boxrightshaded}

\begin{boxleft}Umrechnung\\
Winkelgrößen $\Leftrightarrow$ Bahngrößen
\end{boxleft}\begin{boxrightshaded}
\begin{align}
a_t		&=\alpha\cdot r\\
\vec{a_t}	&=\vec{\alpha} \times \vec{r}\\
v		&=\omega\cdot r\\
\vec{v}		&=\vec{\omega}\times\vec{r}\\
s		&=\varphi\cdot r
\end{align}
\end{boxrightshaded}