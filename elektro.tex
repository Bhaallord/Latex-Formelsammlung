\chapter{Elektrotechnik}
\section{Grundgrößen}

\begin{boxleft}Elementarladung
\end{boxleft}\begin{boxrightshaded}
\begin{align}
e\approx 1,6\cdot 10^{-19}C
\end{align}
\end{boxrightshaded}

\begin{boxleft}ele. Ladung
\end{boxleft}\begin{boxrightshaded}
\begin{align}
\left[Q\right]&=1C=1As\\
Q&=n\cdot e
\end{align}
\end{boxrightshaded}

\begin{boxleft}ele. Strom
\end{boxleft}\begin{boxrightshaded}
\begin{align}
\left[I\right]&=1A\\
i(t)&=\frac{\diff Q}{\diff t}
\end{align}
\end{boxrightshaded}

\begin{boxleft}ele. Stromdichte
\end{boxleft}\begin{boxrightshaded}
\begin{align}
\left[J\right]&=1\frac{A}{mm^2}\\
\vec{J}&=\frac{I}{\vec{A}}
\end{align}
\end{boxrightshaded}

\begin{boxleft}ele. Potenzial
\end{boxleft}\begin{boxrightshaded}
\begin{align}
\left[\varphi\right]&=1V=1\frac{Nm}{As}=1\frac{kgm^2}{As^3}\\
\varphi&=\frac{W}{Q}
\end{align}
\end{boxrightshaded}

\begin{boxleft}ele. Spannung
\end{boxleft}\begin{boxrightshaded}
\begin{align}
\left[U\right]&=1V\\
U_{AB}&=\varphi_a-\varphi_b
\end{align}
\end{boxrightshaded}

\begin{boxleft}ele. Widerstand
\end{boxleft}\begin{boxrightshaded}
\begin{align}
\left[R\right]&=1\Omega=1\frac{V}{A}\\
R&=\frac{U}{I}\\
&=\rho\frac{l}{A}=\frac{1}{\kappa}\frac{l}{A}
\end{align}
\end{boxrightshaded}

\begin{boxleft}ele. Leitwert
\end{boxleft}\begin{boxrightshaded}
\begin{align}
\left[G\right]&=1S=1\frac{A}{V}\\
G&=\frac{I}{U}\\
&=\frac{1}{R}\\
&=\kappa\frac{A}{l}=\frac{1}{\rho}\frac{A}{l}
\end{align}
\end{boxrightshaded}

\begin{boxleft}Temperaturabhängigkeit von Widerstand
\end{boxleft}\begin{boxrightshaded}
\begin{align}
R_2=R_1\cdot\left(1+\alpha\left(\vartheta_2-\vartheta_1\right)+\beta\left(\vartheta_2-\vartheta_1\right)^2\right)
\end{align}
\end{boxrightshaded}

\begin{boxleft}Leistung
\end{boxleft}\begin{boxrightshaded}
\begin{align}
\left[P\right]&=1W=1VA\\
P&=u(t)\cdot i(t)
\end{align}
\end{boxrightshaded}

\begin{boxleft}Mittlere Leistung
\end{boxleft}\begin{boxrightshaded}
\begin{align}
P&=\frac{1}{T}\int_0^T u(t)\cdot i(t)\diff t 
\end{align}
\end{boxrightshaded}

\section{Lineare Quellen}
\begin{boxleft}Lineare Spanungsquelle
\end{boxleft}\begin{boxrightshaded}
\begin{align}
U&=U_q-R_i\cdot I\\
I_K&=\frac{U_q}{R_i}
\end{align}
\end{boxrightshaded}

\begin{boxleft}Lineare Stromquelle
\end{boxleft}\begin{boxrightshaded}
\begin{align}
I&=I_q-\frac{U}{R_i}\\
U_l&=I_q\cdot R_i
\end{align}
\end{boxrightshaded}

\section{Kirchhoffsche Gesetze}


\begin{boxleft}Knotenpunktsatz
\end{boxleft}\begin{boxrightshaded}
\begin{align}
\sum_{i=1}^n I_i=0
\end{align}
\end{boxrightshaded}

\begin{boxleft}Maschensatz
\end{boxleft}\begin{boxrightshaded}
\begin{align}
\sum_{i=1}^n U_i=0
\end{align}
\end{boxrightshaded}

