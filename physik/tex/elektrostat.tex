\section{Elektrisches Feld}

\begin{boxleft}\bla{Ladung}
\des[\ampere\second]{Q}{Ladung}\\
\des[\ampere]{i}{Strom}
\end{boxleft}\begin{boxrightshaded}
\begin{align*}
Q	&=n\cdot e_0\\
	&=CU\\
	&=\int i \diff t
\end{align*}
\end{boxrightshaded}

\begin{boxleft}\bla{Coulombsches Gesetz}
\des[\newton]{F}{Kraft}\\
\des[\volt\per\meter]{E}{Feldstärke}\\
\des[\ampere\second\per\volt\per\meter]{\epsilon}{Elektrische Feldkonstante}\\
\des[\meter]{r}{Abstand der Ladungen}
\end{boxleft}\begin{boxrightshaded}
\begin{align*}
\vv{F}_{12}&=\frac{1}{4\pi\epsilon}\frac{Q_1Q_2}{r^2}\vv{r}_{12}\\
\vv{F}_{12}&=\vv{E}Q\\
\vv{E}&=\frac{1}{4\pi\epsilon}\frac{Q}{r^2}\vv{r}\\
\vv{E}&=-\grad\varphi=-\left(\frac{\partial \varphi}{\partial x}\vv{e}_x+\frac{\partial \varphi}{\partial y}\vv{e}_y+\frac{\partial \varphi}{\partial z}\vv{e}_z\right)
\end{align*}
\end{boxrightshaded}

\begin{boxleft}\bla{Feldstärke mehrere Punktladung}
\end{boxleft}\begin{boxrightshaded}
\begin{align*}
\vv{E}(\vv{r})&=\sum_{i=1}^{N}\vv{E}_i{\vv{r}_i}
\end{align*}
\end{boxrightshaded}

\begin{boxleft}\bla{Spannung}
\des[\volt]{U}{ele. Spannung}\\
\des[\volt]{\varphi}{ele. Potenzial}
\end{boxleft}\begin{boxrightshaded}
\begin{align*}
\varphi_A&=-\int_\infty^A\vv{E}\circ\diff s\\
\varphi(r)&=\frac{Q}{4\pi \epsilon r}&&\text{Kugel}\\
U_{AB}&=\frac{W_{AB}}{Q}\\
U_{AB}&=-\int_A^B\vv{E}\circ\diff\vv{s}\\
U_{AB}&=\oint_s\vv{E}\circ\diff\vv{s}=0\\
U_{AB}&=\varphi_A-\varphi_B=-\int_\infty^A\vv{E}\circ\diff\vv{s}-\left(-\int_\infty^B\vv{E}\circ\diff\vv{s}\right)
\end{align*}
\end{boxrightshaded}

\subsection{Elektrostatik}

\begin{boxleft}\bla{Elektrischer Fluss, Verschiebungsfluss}
\des[\volt\meter]{\psi}{Elektrischer Fluss}
\end{boxleft}\begin{boxrightshaded}
\begin{align*}
\psi&=\int_A\vv{E}\circ\diff\vv{A}\\
\psi&=\oint_A\vv{E}\circ\diff\vv{A}=\frac{Q}{\epsilon}\\
\end{align*}
\end{boxrightshaded}

\begin{boxleft}\bla{Elektrische Flussdichte}
\des[\ampere\second\per\meter\tothe{2}]{D}{Elektrische Flussdichte}
\end{boxleft}\begin{boxrightshaded}
\begin{align*}
\vv{D}&=\frac{\diff Q}{\diff A}\vv{e}_A\\
\vv{D}&=\epsilon\vv{E}\\
Q&=\oint_AD\diff A
\end{align*}
\end{boxrightshaded}


\begin{boxleft}\bla{Arbeit im ele. Feld}
\des[\ampere\second\per\volt]{C}{Kapazität}\\
\des[\joule]{W}{Arbeit}\\
\des[\joule\per\meter\tothe{3}]{w}{Energiedichte}
\end{boxleft}\begin{boxrightshaded}
\begin{align*}
w&=\frac{1}{2}\vv{E}\circ\vv{D}\\
W &=\int_Vw\diff V\\
  &=-Q\int_A^B\vv{E}\circ\diff\vv{s}\\
  &=\int_U Q\diff U= \int_U CU \diff U=\frac{1}{2}CU^2
\end{align*}
\end{boxrightshaded}

\subsection{Elektrodynamik}

\begin{boxleft}\bla{Kapazität}
\des[\ampere\second\per\volt]{C}{Kapazität}
\end{boxleft}\begin{boxrightshaded}
\begin{align*}
Q&=CU
\end{align*}
\end{boxrightshaded}

\begin{boxleft}\bla{Ohmsches Gesetz}
\des[\ampere\per\meter\tothe{2}]{j}{Stromstärke}\\
\des[\joule]{W}{Arbeit}\\
\des[\joule\per\meter\tothe{3}]{w}{Energiedichte}
\end{boxleft}\begin{boxrightshaded}
\begin{align*}
I &=\oint_A\vv{j}\circ\diff\vv{A}\\
  &=\oint_A \kappa\vv{E}\circ\diff\vv{A}\\
  &=\kappa E\cdot 4\pi r^2&&\text{Kugel}
\end{align*}
\end{boxrightshaded}


