
\section{Thermodynamik}

\subsection{Dehnung}

\begin{boxleft}\bla{Wärmedehnung}
\des[\per\kelvin]{\beta}{Dichteausdehnungskoeffizient}\\
\des[\per\kelvin]{\gamma}{Volumenausdehnungskoeffizient}\\
\des[\per\kelvin]{\alpha}{Längenausdehnungskoeffizient}\\
\des[\kilo\gram\per\meter\tothe{3}]{\rho}{Dichte}\\
\des[\meter\tothe{3}]{V}{Volumen}\\
\des[\meter]{l}{Länge}\\
\des[\kelvin]{T}{Temperatur}\\
\des[\kelvin]{T_0}{Ausgangstemperatur}
\end{boxleft}\begin{boxrightshaded}
\begin{align*}
\rho(T)&=\rho_0(1-\beta(T-T_0))\\
V(T)&=V_0(1+\gamma(T-T_0))\\
l(T)&=l_0(1+\alpha(T-T_0))\\
\gamma&\approx 3\cdot \alpha\\
\gamma&\approx \beta
\end{align*}
\end{boxrightshaded}

\subsection{Wärme}

\begin{boxleft}\bla{Wärme}
\des[\joule]{Q}{Wärme}\\
\des[\joule\per\kelvin\per\kilo\gram]{c}{spez. Wärmekapazität}\\
\des[\joule\per\kelvin]{C}{Wärmekapazität}\\
\des[\joule\per\kelvin\per\mole]{c_{mol}}{molare Wärmekapazität}\\
\des[\mole]{n}{Stoffmenge}
\end{boxleft}\begin{boxrightshaded}
\begin{align*}
\Delta Q&=c\cdot m(T-T_0)\\
\Delta Q&=C(T-T_0)\\
\Delta Q&=\int_{T_0}^T c\cdot m \diff T\\
\Delta Q&=c_{mol}\cdot n(T-T_0)
\end{align*}
\end{boxrightshaded}

\begin{boxleft}\bla{Mischtemperatur}
\end{boxleft}\begin{boxrightshaded}
\begin{align*}
T_m&=\frac{\sum_{i=1}^n T_i m_i c_i}{\sum_{i=1}^n m_i c_i}
\end{align*}
\end{boxrightshaded}

\begin{boxshaded}
\begin{align*}
&\dot{Q} \quad\text{Ist durch einen mehrschichtiges stationäres System Konstant}
\end{align*}
\end{boxshaded}

\begin{boxleft}\bla{Wärmeleitung}
\des[\watt]{\dot{Q}}{Wärmestrom}\\
\des[\watt\per\metre\tothe{2}]{\vv{\dot{q}}}{Wärmestromdichte}\\
\des[\metre\tothe{2}]{A}{Fläche}\\
\des[\watt\per\metre\per\kelvin]{\lambda}{Wärmeleitzahl}\\
\des[\metre]{s}{Dicke der $\lambda$ Schicht}
\end{boxleft}\begin{boxrightshaded}
\begin{align*}
\dot{Q}&=\frac{\diff Q}{\diff t}=\varPhi=P\\
\vv{\dot{q}}&=\frac{\dot{Q}}{A}\cdot\vv{e_A}\\
\vv{\dot{q}}&=-\lambda\grad{T}\\
\vv{\dot{q}}&=\frac{\lambda}{s}\left(T_A-T_B\right)\cdot\vv{e_s}\\
\dot{q}&=\frac{1}{\sum_{i=1}^n\frac{s_i}{\lambda_i}}\cdot\left(T_A-T_B\right)
\end{align*}
\end{boxrightshaded}


\begin{boxleft}\bla{Wärmekonvektion}
\des[\watt\per\metre\tothe{2}\per\kelvin]{\alpha}{Wärmeübergangszahl}
\end{boxleft}\begin{boxrightshaded}
\begin{align*}
\dot{q}&=\alpha\left(T_A-T_B\right)\\
\dot{q}&=\frac{1}{\sum_{i=1}^n\frac{1}{\alpha_i}}\cdot\left(T_A-T_B\right)
\end{align*}
\end{boxrightshaded}

\begin{boxleft}\bla{Wärmewiderstand}
\des[\kelvin\per\watt]{R_{th}}{Wärmewiderstand}
\end{boxleft}\begin{boxrightshaded}
\begin{align*}
R_{th}&=\frac{T_A-T_B}{\dot{q}\cdot A}\\
R_{th}&=\frac{s}{\lambda A}\\
R_{th}&=\frac{1}{\alpha A}\\
R_{th}&=\sum_{i=1}^n R_{i}
\end{align*}
\end{boxrightshaded}

\begin{boxleft}\bla{Wärmeübertragung}
\des[\watt\per\kelvin\per\metre\tothe{2}]{k}{Wärmedurchgangszahl}
\end{boxleft}\begin{boxrightshaded}
\begin{align*}
k&=\frac{1}{\sum_{i=1}^n\frac{s_i}{\lambda_i}+\sum_{i=1}^n\frac{1}{\alpha_i}+\sum_{i=1}^n  A_{i}\cdot R_{i}}\\
\dot{q}&=\frac{1}{\sum_{i=1}^n\frac{s_i}{\lambda_i}+\sum_{i=1}^n\frac{1}{\alpha_i}+\sum_{i=1}^n A_{i}\cdot R_{i}}\cdot\left(T_A-T_B\right)\\
\dot{q}&=k\cdot\left(T_A-T_B\right)
\end{align*}
\end{boxrightshaded}

\begin{boxleft}\bla{Wärmestrahlung}
\des{\varepsilon}{Emissionsgrad}\\
\des[\watt\per\metre\tothe{2}\per\kelvin\tothe{4}]{\sigma}{Stefan-Boltzmann-Konstante}\\
\des[\watt\per\metre\tothe{2}\per\kelvin]{C}{Strahlungsaustauschkonstante}\\
\des{\alpha}{Absorptionsgard}\\
\des{\tau}{Transmissionsgard}\\
\des{\vartheta}{Reflexionsgard}\\
\end{boxleft}\begin{boxrightshaded}
\begin{align*}
\alpha&=\varepsilon\\
1&=\alpha+\tau+\vartheta\\
\dot{Q}&=\varepsilon A \sigma T^4\\
\sigma&=5,6704\cdot10^{-8}\si{\watt\per\metre\tothe{2}\per\kelvin\tothe{4}}\\
\dot{Q}_{AB}&=C_{AB}A_A\left(T_A^4-T_B^4\right)\\
C_{AB}&=\varepsilon_{AB}\sigma=\frac{\sigma}{\frac{1}{\varepsilon_A}+\frac{1}{\varepsilon_B}-1}=\frac{1}{\frac{1}{\sigma_A}+\frac{1}{\sigma_B}-\frac{1}{\sigma}}&&\text{Parallel}\\
C_{AB}&=\frac{\sigma}{\frac{1}{\varepsilon_A}+\frac{A_A}{A_B}\left(\frac{1}{\varepsilon_B}-1\right)}&&\text{$A_A$ von $A_B$ umschlossen}\\
C_{AB}&\approx\varepsilon_A\sigma&&\text{parallel ($A_A\ll A_B$)}
\end{align*}
\end{boxrightshaded}


\begin{boxleft}\bla{Wiensches Verschiebungsgesetz}
\destext{Gibt das Maximum der Wellenlänge zur Temperatur an}\\
\des[\metre\kelvin]{b}{Wiensche Konstante}
\end{boxleft}\begin{boxrightshaded}
\begin{align*}
\lambda_{max}&=\frac{b}{T}\\
b&=2,8978\cdot10^{-3}\si{\metre\kelvin}
\end{align*}
\end{boxrightshaded}

\begin{boxleft}\bla{Strahler}
\destext{Berechnug der Strahlungsleistung in einen Bereich $\lambda$}\\
\des[\watt]{\Phi}{Strahlungsleistung}\\
\des[\watt\per\steradian]{I}{Strahlstärke}\\
\des[\watt\per\metre\tothe{2}\per\steradian]{I}{Strahldichte}\\
\des[\watt\per\metre\tothe{2}]{M}{spez. Ausstrahlung}\\
\des[\steradian]{\Omega}{Raumwinkel}
\end{boxleft}\begin{boxrightshaded}
\begin{align*}
\Phi_e&=\frac{\diff Q}{\diff t}\\
I_e&=\frac{\diff \Phi_e}{\diff \Omega}\\
M_e&=\frac{\diff \Phi_e}{\diff A}\\
L_e&=\frac{\diff I_e}{\diff A}\\
\Omega&=\frac{A}{r^2}
\end{align*}
\end{boxrightshaded}

\begin{boxleft}\bla{Schwarzer Strahler(Plancksches Strahlungsgesetz)}
\destext{Berechnug der Strahlungsleistung in einen Bereich $\lambda$}\\
\des[\joule\per\kelvin]{k}{Boltzmann Konstante}\\
\des[\joule\second]{h}{Planksches Wirkungsquantum}\\
\des[\metre]{\lambda}{Wellenlänge}\\
\des[\metre\per\second]{c}{Lichtgeschwindigkeit}\\
\des{n}{Brechungszahl}
\end{boxleft}\begin{boxrightshaded}
\begin{align*}
\varepsilon&=\alpha=1\\
P_\lambda&=\frac{\diff P}{\diff \lambda}=\frac{2\pi h c^2}{\lambda^5}\cdot\frac{A}{e^{\frac{hc}{k\lambda T}}-1}\\
M_{e,\lambda}&=\frac{\diff P_\lambda}{\diff A}=\frac{c_1}{n^2\lambda^5\cdot\left(e^{\frac{c_2}{n\lambda T}}-1\right)}\\
L_{e,\lambda}&=\frac{M_{e,\lambda}}{\pi\Omega}\\
c_1&=2\pi h c^2=3,7418\cdot10^{-16}\si{\watt\per\metre\tothe{2}}\\
c_2&=\frac{h c}{k}=0,01439\si{\metre\kelvin}
\end{align*}
\end{boxrightshaded}

\subsection{Zustandsänderung des idealen Gases}


\begin{boxshaded}
\begin{align*}
&\text{Ideales Gas bedeutet das die Teilchen nicht in Wechselwirkung geraten, sie kein Volumen und es kommt zu keinen Phasenübergang}
\end{align*}
\end{boxshaded}

\begin{boxleft}\bla{Energie}
\des[\joule]{H}{Enthalpie}\\
\des[\joule\per\kelvin]{c_p}{spez. Wärmekapazität(p=const)}
\end{boxleft}\begin{boxrightshaded}
\begin{align*}
U_{12}&=Q_{12}+W_{12}\\
\diff H&=c_pm\diff T=U+p\diff V&&\text{Nur Isobar}\\
\diff S&=\frac{\diff Q}{T}
\end{align*}
\end{boxrightshaded}

\begin{boxleft}\bla{Zustandsgleichung}
\des[\pascal]{p}{Druck}\\
\des[\kelvin]{T}{Temperatur}\\
\des[\joule\per\kelvin]{k}{Boltzmann-Konstante}\\
\des{N}{Teilchenanzahl}\\
\des[\meter\tothe{3}]{V}{Volumen}\\
\des[\joule\per\kilo\gram\per\kelvin]{R_s}{spez. Gaskonstante}\\
\des[\joule\per\kilo\gram\per\kelvin]{R}{Gaskonstante}\\
\des[\mole]{n}{Stoffmenge}\\
\des[\joule\per\kelvin]{c_V}{spez. Wärmekapazität(V=const)}
\end{boxleft}\begin{boxrightshaded}
\begin{align*}
\frac{pV}{T}&=\text{const}\\
pV&=NkT\\
pV&=mR_sT\\
pV&=nRT\\
R_s&=\frac{nR}{m}\\
R_s&=c_p-c_v
\end{align*}
\end{boxrightshaded}


\begin{boxleft}\bla{Isotherm}
\des[\joule]{\Delta U}{Innere Energie}\\
\des[\joule\per\kelvin]{S}{Innere Energie}\\
\des[\joule\per\kelvin]{c_V}{spez. Wärmekapazität(V=const)}\\
\des[\joule\per\kelvin]{c_p}{spez. Wärmekapazität(p=const)}
\end{boxleft}\begin{boxrightshaded}
\begin{align*}
pV&=\text{const}\\
T&=\text{const}\\
U_{12}&=0\\
U_{12}&=Q_{12}+ W_{12}\\
Q_{12}&=-W_{12}\\
W_{12}&=p_1V_1\ln{\frac{V_2}{V_1}}\\
W_{12}&=p_1V_1\ln{\frac{p_1}{p_2}}\\
S_{12}&=mc_p\ln{\frac{V_2}{V_1}}+mc_V\ln{\frac{p_2}{p_1}}
\end{align*}
\end{boxrightshaded}

\begin{boxleft}\bla{Isobarer}
\end{boxleft}\begin{boxrightshaded}
\begin{align*}
\frac{V}{T}&=\text{const}\\
p&=\text{const}\\
Q_{12}&=mc_p\left(T_2-T_1\right)\\
W_{12}&=-p\left(V_2-V_1\right)\\
U_{12}&=Q_{12}+ W_{12}\\
S_{12}&=mc_p\ln{\frac{V_2}{V_1}}
\end{align*}
\end{boxrightshaded}

\begin{boxleft}\bla{Isochor}
\end{boxleft}\begin{boxrightshaded}
\begin{align*}
\frac{p}{T}&=\text{const}\\
V&=\text{const}\\
Q_{12}&=mc_v\left(T_2-T_1\right)\\
W_{12}&=0\\
U_{12}&=Q_{12}\\
S_{12}&=mc_v\ln{\frac{p_2}{p_1}}
\end{align*}
\end{boxrightshaded}



\begin{boxleft}\bla{Adiabat}
\end{boxleft}\begin{boxrightshaded}
\begin{align*}
pV^\kappa&=\text{const}\\
Q&=\text{const}\\
\kappa&=\frac{c_p}{c_V}\\
\frac{T_2}{T_1}&=\left(\frac{V_2}{V_1}\right)^{1-\kappa}=\left(\frac{p_2}{p_1}\right)^{\frac{\kappa-1}{\kappa}}\\
Q_{12}&=0\\
W_{12}&=mc_v\left(T_2-T_1\right)\\
W_{12}&=\frac{RT_1}{\kappa-1}\left(\left(\frac{V_2}{V_1}\right)^{1-\kappa}-1\right)\\
U_{12}&=W_{12}\\
S_{12}&=0;
\end{align*}
\end{boxrightshaded}


\begin{boxleft}\bla{Kreisprozess}
\end{boxleft}\begin{boxrightshaded}
\begin{align*}
\oint \diff U&=0\\
\oint \diff U&= \oint \diff Q +\oint \diff W\\
\oint \diff S&=0 &&\text{Revesiebel}\\
\oint \diff S&>0 &&\text{Irrevesiebel}\\
\end{align*}
\end{boxrightshaded}


\begin{boxleft}\bla{Carnot}
\des{\eta_C}{Carnot Wirkungsgrad}
\end{boxleft}\begin{boxrightshaded}
\begin{align*}
\eta_C&=\frac{W_{ab}}{Q_{zu}}\\
\eta_C&=\frac{Q_{zu}-Q_{AB}}{Q_{zu}}\\
\eta_C&=\frac{T_h-T_n}{T_n}
\end{align*}
\end{boxrightshaded}